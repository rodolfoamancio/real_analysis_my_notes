\chapter{Sequences and series}

\section{The starting problem}

Basically a series is a sum of infinite terms. On the following example, some problems will appear as we try to manipulate the series as standard mathematical entities. Consider, for instance:

\begin{equation}
    \sum \limits_{n=1}^\infty \frac{(-1)^{n+1}}{n} = 1 - \frac{1}{2} + \frac{1}{3} - \frac{1}{4} + ...
\end{equation}

We can consider the partial sum, $s_n$, \emph{i.e.} the sum of the $n$ first terms of the series. In this case we would obtain: $s_1 = 1$, $s_2 = 1/2$, $s_3 = 5/6$,... and so on. Interestingly, the odd sums decrease ($s_1 > s_3 > s_5 > ...$), while the even sums increase ($s_2 < s_4 < s_6 < ...$). It gives the idea that $\{s_n\}$ converges to some number $S$. And we may feel tempted to write:

\begin{equation*}
    S = 1 - \frac{1}{2} + \frac{1}{3} - \frac{1}{4} + ...
\end{equation*}

However, the use of standard mathematical notation ($+, -, =$) for series can be misleading. Take the previous equation, multiply it for $1/2$ and add it to itself. We would get:

\begin{equation*}
    \frac{3}{2}S = 1 - \frac{1}{2} + \frac{1}{3} - \frac{1}{4} + ...
\end{equation*}

Which seems to be a contradiction to our initial claim. In a certain sense, addition in this infinite setting is not commutative.

Another example is the series:

\begin{equation}
    \sum \limits_{n=1}^\infty (-1)^n = -1 + 1 - 1 + 1 - 1 + ...
\end{equation}

Depending on how we group the terms we would find different results:

\begin{equation*}
    (-1 + 1) + (-1 + 1) + (-1 + 1) + ... = 0
\end{equation*}

On the other hand,

\begin{equation*}
    -1 + (1-1) + (1-1) + (1-1) + ... = 1
\end{equation*}

In order to deal with the tricks hidden in infinite series, we begin by discussing sequences.

\section{Sequences}

\subsection{Convergent sequences}

\begin{definition}[Sequence]
    A sequence is a function, $f$, whose domain is $\N$. In this way, $f: \N to \R$. Hence, $f(n)$ is the $n$-th term of the sequence. Notation: usually, a sequence is presented in the form $\{ x_n \}$, or $\{ x_n \}_{n=1}^\infty$, or $ x_1, x_2, x_3, ...$.
\end{definition}

\begin{definition}[Convergence of a sequence]
    A sequence $\{x_n \}$ converges to $x$ if $\forall \varepsilon > 0, \exists N \in N$ such that $|x_n - x| < \varepsilon, \forall n \geq N$. There are a few different ways to denote convergence, such as $\{ x_n\} \to x$, $\lim \limits_{n \to \infty} x_n = x$ or $x_n \to x$.
\end{definition}

The negation of the convergence of a sequence would be:

\begin{definition}
    A sequence  $\{ x_n\}$ does not converge to $x$ if $\exists \varepsilon_0 > 0$ such that $\exists m \in \N$ such that $|x_n - x| \geq \varepsilon, \forall n \geq m$.
\end{definition}

\paragraph{Example}

\begin{equation*}
    \lim \limits_{n \to \infty} \frac{1}{n^2 + 30n + 1} = 0
\end{equation*}

\begin{proof}
    We need to find $N \in \N$ such that
    \begin{equation*}
        \frac{1}{n^2 + 30n + 1} < \varepsilon, \forall n \geq N
    \end{equation*}
    But
    \begin{equation*}
        \frac{1}{n^2 + 30n + 1} \leq \frac{1}{n^2 + 30n} \leq \frac{1}{30n} \leq \frac{1}{n}
    \end{equation*}
    Hence, if $1/n < \varepsilon$ then we would obtain the initial inequality. \\
    Let $\varepsilon > 0$, set $N \in \N$ such that $1/N < \varepsilon$. Then, for all $n \geq N$:
    \begin{equation*}
        \left| \frac{1}{n^2+30n+1} - 0\right| = \frac{1}{n^2+30n+1} \leq  \frac{1}{30n} \leq \frac{1}{n} \leq \frac{1}{N} < \varepsilon
    \end{equation*}
\end{proof}

\begin{definition}[Bounded sequences]
    A sequence $\{x_n\}$ is bounded if there exists a number $M > 0$ such that $|x_n| < M, \forall n \in \N$.
\end{definition}

\begin{theorem}
    If $\{ x_n\}$ is convergent, then $\{ x_n\}$ is bounded.
\end{theorem}

\begin{proof}
    Suppose $\{ x_n\} \to x$, then $\exists N \in \N$ such that $\forall n \geq N, |x_n - x| < \varepsilon, \forall \varepsilon > 0$. Regardless if $x$ is positive or negative, we can write $|x_n| < |x| + \varepsilon$. Define $M = \max \{||x_1|, |x_2|, ..., |x_{N-1}|, |x| + \varepsilon \}$. Then, $|x_n| \leq M, \forall n \in \N$. 
\end{proof}

\begin{definition}
    A sequence $\{ x_n \}$ is:
    \begin{enumerate}
        \item Monotone increasing, if $x_n \leq x_{n+1}, \forall n \in \N$,
        \item Monotone decreasing, if $x_n \geq x_{n+1}, \forall n \in \N$,
        \item If it is either monotone increasing or decreasing, then it is called monotone.
    \end{enumerate}
\end{definition}

\begin{theorem}
    A monotonic sequence is convergent if, and only if, it is bounded.
\end{theorem}

\begin{proof}
    Suppose $\{ x_n\}$ is a monotonic increasing sequence. Then,
    \begin{enumerate}
        \item $( \Rightarrow )$ follows from the previous theorem.
        \item $( \Leftarrow )$. Suppose $ \{x_n \} $ is bounded. Then, $ \{ x_n: n \in \N\} \subseteq \R$ is a bounded set. Let $x = \sup \{ x_n: n \in \R\}$. We claim 
        \begin{equation*}
            x = \lim \limits_{n\to \infty} x_n
        \end{equation*}
        Let $\varepsilon > 0$. Since $x - \varepsilon$ is not an upper bound for $\{ x_n: n \in \N\}$, $\exists M_0 \in \N$ such that $x_n - \varepsilon < x_{M_0} < x$. Choose $M = M_0$, then $\forall n \geq M, x-\varepsilon < x_{M_0} < x_n \leq x + \varepsilon$, or $x-\varepsilon < x_M < x+\varepsilon$.
    \end{enumerate}
\end{proof}

\begin{theorem}[Algebraic limit theorem]
    Let $\{ a_n\} \to a$ and $\{ b_n\} to b$. Then,
    \begin{enumerate}
        \item $\{ ca_n\} \to ca, \forall c \in \R$
        \item $\{ a_n + b_n\} \to a + b$
        \item $\{ a_nb_n\} \to ab $
        \item $\{a_n/b_n\} \to a/b$, given $b \ne 0$
    \end{enumerate}
\end{theorem}

\begin{proof}
    Let's take each item individually:
    \begin{enumerate}
        \item First, note $|ca_n - ca| = |c||a_n - a|$. Hence, for $\varepsilon > 0$ we have $|ca_n - ca| < \varepsilon \Leftrightarrow |a_n-a| < \varepsilon/|c|$. Since $\{ a_n \} \to a$ then $\exists N \in \N$ such that $|a_n - a| < \varepsilon / |c|$, so $|ca_n - ca| = |c||a_n-a| < |c|\varepsilon/|c|, \forall n \geq N$.
        \item From the triangle inequality, $|(a_n+b_n)-(a-b)| \leq |a_n-a|+|b_n-b|$. Set $N_1 \in \N$ such that $|a_n-a|< \varepsilon/2, \forall n \geq N_1$ with $\varepsilon > 0$. And set $N_2 \in \N$ such that $|b_n-b| < \varepsilon/2, \forall n \geq N_2$. Then, for $N = \max \{ N_1, N_2\}$ we obtain: $|(a_n+b_n)-(a+b)| \leq |a_n-a| + |b_n-b| < \varepsilon/2 + \varepsilon/2 = \varepsilon$.
        \item First, $|a_nb_n - ab| = |a_nb_n -ab_n + ab_n - ab| \leq |a_nb_n-ab_n| + |ab_n - ab| = |b_n||a_n-a| + |a||b_n-b|$. Take $N_1 \in \N$ such that $|b_n-b| < \varepsilon/(2|a|), \forall n \geq N_1$ with $\varepsilon > 0$. Since every convergent sequence is bounded, take $M > 0$ so that $|b_n| < M, \forall n \in \N$. Then, set $N_2 \in \N$ such that $|a_n-a| < \varepsilon/(2M), \forall n \geq N_2$. Finally, for $N = \max \{ N_1, N_2\}$ we obtain $|a_nb_n - ab| \leq |b_n||a_n-a| + |a||b_n-b| < M\varepsilon/(2M) + |a|\varepsilon/(2|a|) = \varepsilon$.
        \item $\{ a_n/b_n\} \to a/b$ follows from the previous result by noting $\{(1/b_n)\} \to 1/b$, provided $b \ne 0$. So,
        \begin{equation*}
            \left |
            \frac{1}{b_n} - \frac{1}{b}
            \right | = \frac{|b -b_n|}{|b||b_n|}
        \end{equation*}
        $|b_n-b|$ can be made arbitrarily small. On the other hand, considering $\varepsilon_0 = |b|/2$, define $N_1 \in \N$ such that $|b_n-b|<|b|/2, \forall n \geq N_1$, hence $|b_n| > |b|/2, \forall n \geq N_1$. Now, set $N_2 \in \N$ such that $|b_n-b| < \varepsilon|b|^2/2$. Taking $N = \max \{ N_1, N_2\}$ leads to
        \begin{equation*}
            \left|
                \frac{1}{b_n} - \frac{1}{b}
                \right| = 
                |b-b_n|\frac{1}{|b||b_n|} < \frac{\varepsilon |b|^2}{2} \frac{1}{|b|\frac{|b|}{2}} = \varepsilon, \forall n \geq N
        \end{equation*}
    \end{enumerate}
\end{proof}

\begin{theorem}[Order limit theorem]
    Assume $\{a_n\} \to a$ and $\{ b_n\} \to b$, then:
    \begin{enumerate}
        \item If $a_n \geq 0, \forall n \in \N \Rightarrow a \geq 0$,
        \item If $a_n \leq b_n, \forall n \in \N \Rightarrow a \leq b$,
        \item If there exists $c \in \R$ such that $c \leq b_n, \forall n \in \N$, then $c \leq b$. Similarly, if $a_n \leq c, \forall n \in \N$ then $a \leq c$.
    \end{enumerate}
\end{theorem}

\begin{proof}
    Proof for each statement:
    \begin{enumerate}
        \item Assume $a < 0$. Consider $\varepsilon  = |a|$, since $\{ a_n\} \to a$ we have $|a_n - a|< |a|, \forall n \geq N$. In particular, $|a_N - a|<|a|$ hence $a_N < 0$ which is a contradiction. Therefore, $a \geq 0$.
        \item From the algebraic theorem,  $\{ b_n - a_n\} \to a-b$. Since $b_n-a_n \geq 0$ from the previous result, we get $b - a \geq 0$.
        \item Take $a_n = c, \forall n \in \N$. From the previous theorem, if $c \leq b_n$ then $b-c \geq 0$. Hence, $b \geq c$.
    \end{enumerate}
\end{proof}

\subsection{Operations involving convergent sequences}

In order to find out if a sequence converges, and to what value, there are a few tools at our disposal. We begin with the very popular Squeeze theorem, sometimes refered here as ST.

\begin{theorem}[Squeeze theorem]
    Let $\{a_n\}, \{b_n\}, \{x_n\}$ be sequences such that $\forall n \in \N$, $a_n \leq x_n \leq b_n$. Suppose $\{a_n\}$ and $\{b_n\}$ both converge and
    \begin{equation*}
        \lim \limits_{n \to \infty} a_n = x = \lim \limits_{n \to \infty} b_n
    \end{equation*}
    So, $\{x_n\} \to x$.
\end{theorem}

\begin{proof}
    Let $\varepsilon > 0$. Since $\{a_n\} \to x$, $\exists M_0 \in \N$ such that $\forall n \geq M_0, |a_n-x| < \varepsilon \Rightarrow x-\varepsilon < a_n$.\\
    Since $\{b_n\} \to x$, then $\exists M_1 \in \N$ such that $\forall n \geq M_1, |b_n-x|<\varepsilon \Rightarrow b_n < x + \varepsilon$. \\
    Choose $M = \max\{ M_0, M_1\}$. Then, if $n \geq M, x-\varepsilon < a_n \leq x_n \leq b_n < x-\varepsilon \Rightarrow |x_n-x|>\varepsilon$.
\end{proof}

The limit of a function can be also expressed in another form, which can be pretty useful.

\begin{theorem}
    Another way to check if $\{x_n\} \to x$ would be
    \begin{equation*}
        \lim \limits_{n \to \infty} x_n = x \Longleftrightarrow \lim \limits_{n \to \infty} |x_n - x| = 0
    \end{equation*}
\end{theorem}

\paragraph{Example}

Show that 

\begin{equation*}
    \lim \limits_{n \to \infty} \frac{n^2}{n^2+n+1} = 1
\end{equation*}

\begin{proof}
    We have
    \begin{equation*}
        \left |
            \frac{n^2}{n^2+n+1} - 1
        \right | = 
        \left |
            \frac{-n-1}{n^2+n+1}
        \right | = 
        \frac{n+1}{n^2+n+1} \leq
        \frac{n+1}{n^2+n} = \frac{1}{n}
    \end{equation*}
    Thus,
    \begin{equation*}
        0 \leq 
        \left |
            \frac{n^2}{n^2+n+1} - 1
        \right |
        \leq
        \frac{1}{n}
        \Longrightarrow
        \lim \limits_{n \to \infty}
        \left |
            \frac{n^2}{n^2+n+1} - 1
        \right | = 0
    \end{equation*}
    by the Squeeze theorem.
\end{proof}

\subsection{Some special sequences}

\begin{theorem}
    If $\{x_n\}$ is a convergent sequence such that $\forall n \in \N, x_n \geq 0$, then $\{\sqrt{x_n}\}$ is convergent and
    \begin{equation}
        \lim \limits_{n \to \infty} \sqrt{x_n} = \sqrt{\lim \limits_{n \to \infty} x_n}
    \end{equation}
\end{theorem}

\begin{proof}
    Let $x = \lim \limits_{n \to infty} x_n$, then:
    \begin{enumerate}
        \item If $x = 0$. Let $\varepsilon > 0$ then, since $\{x_n\} \to 0$, $\exists N_0 \in \N$ such that $x_n = |x_n-0| \varepsilon^2, \forall n \geq N_0$. Choose $N = N_0$, then $\sqrt{x_n}-\sqrt{0} = \sqrt{x_n} < \sqrt{\varepsilon^2} = \varepsilon$
        \item If $ x > 0$. Then, 
        
        \begin{align}
            |\sqrt{x_n} - \sqrt{x}| &= \left |
                \frac{\sqrt{x_n}-\sqrt{x}}{\sqrt{x_n}+\sqrt{x}}(\sqrt{x_n}+\sqrt{x})
            \right | \\
            &= \frac{1}{\sqrt{x_n} - \sqrt{x}}|x_n-x| \\
            &\leq \frac{1}{\sqrt{x}}|x_n-x|, \forall n \in \N
        \end{align}
        
        And,
        \begin{equation*}
            0 \leq |\sqrt{x_n}-\sqrt{x}| \leq \frac{1}{\sqrt{x}}|x_n-x|, \forall n \in \N
        \end{equation*}
        So, by the squeeze theorem, $\lim \limits_{n\to\infty} |\sqrt{x_n}-\sqrt{x}| = 0$
    \end{enumerate}
\end{proof}

\begin{theorem}
    If $\{x_n\}$ is convergent and $\lim \limits_{n\to\infty} x_n = x$, then $\{|x_n|\}$ is convergent and $\lim \limits_{n \to \infty} |x_n| = |x|$.
\end{theorem}

\begin{proof}
    Note that $|x| = \sqrt{x^2}, \forall x \in \R$. Then,
    \begin{equation*}
        \lim \limits_{n\to \infty} |x_n| = \lim \limits_{n\to \infty} \sqrt{x_n^2} = \sqrt{x^2} = x
    \end{equation*}
\end{proof}

\begin{theorem}
    If $x \in (0,1)$ then $\lim \limits_{n \to \infty} c^n = 0$. If $c > 1$ then $\{c^n\}$ is unbounded.
\end{theorem}

\begin{proof}
    For each case:
    \begin{enumerate}
        \item If $0 < c < 1$. Note that $0 < c^{n+1} < c^n < 1, \forall n \in \N$. This can be shown by induction:
        \begin{itemize}
            \item Base case, consider $0 < c^2 < c < 1$, since $0 < c < 1$.
            \item Inductive case, consider $0 < c^{m+1} < c^m < 1$ to be true. Multiplying the former inequality by $c$ we obtain $0 < c^{m+2} < c^{m+1}$.
        \end{itemize}
        Thus, $\{c^n\}$ is monotone decreasing sequence and is bounded below, which implies $\{c^n\}$ is convergent. Set $L = \lim \limits_{n \to \infty} c^n$. Take $\varepsilon > 0$, then $\exists N \in \N$ such that
        
        \begin{align*}
            (1-c)|L| &= |L-cL| = |L - c^{M+1} + c^{M+1} - cL| \\
            &\leq |L-c^{M+1}| + c|c^M-L| \\
            &< (1-c)\frac{\varepsilon}{2} + c(1-c)\frac{\varepsilon}{2} \\
            &< (1-c) \varepsilon, \forall n \geq N
        \end{align*}
        
        Hence, $|L| < \varepsilon, \forall \varepsilon>0 \longrightarrow L = 0$.
        \item For $c > 1$. Note that $\forall B \geq 0, \exists n \in \N$ such that $c^n > B$. For $n \in \N$ such that $n > B/(c-1)$ then $c^n = (1+(1-c))^n \geq 1 + n(c-1) \geq n(c-1) > B$. Hence, $\{c^n\}$ is unbounded $\forall c > 1$ and the sequence does not converge.
    \end{enumerate}
\end{proof}

\begin{theorem}
    If $p>0$ then $\lim \limits_{n \to \infty}n^{-p} = 0$
\end{theorem}

\begin{proof}
    Let $\varepsilon > 0$. Take $N > (1/\varepsilon)^{1/p}$, then
    \begin{equation*}
        \left |
        \frac{1}{n^p} - 0
        \right | = 
        \frac{1}{|n^p|} \leq
        \frac{1}{N^p} <
        \varepsilon
    \end{equation*}
\end{proof}

\begin{theorem}
    If $p>0$ then $\lim \limits_{n \to \infty} p^{1/n}=1$.
\end{theorem}

\subsection{Subsequences and Bolzano–Weierstrass theorem}

\begin{definition}[Subsequence]
    Let $\{x_n\}$ be a sequence of real numbers, and $\{n_k\}$ be a strictly increasing sequence of natural numbers. Then, $\{x_{n_k}\}_{k=1}^\infty$ is called a subsequence of $\{x_n\}$.
\end{definition}

\begin{theorem}
    If $\{x_n\}$ converges to $x$ then any subsequence of $\{x_n\}$ will converge to $x$.
\end{theorem}

\begin{proof}
    Suppose $\{x_n\} \to x$. Let $\varepsilon > 0$, then $\exists M_0 \in \N$ such that $\forall n \geq M_0, |x_n-x| < \varepsilon$. Choose, $M=M_0$. If $k \geq M$, then $n_k \geq k \geq M = M_0$, hence for $\varepsilon>0, \exists M \in \N$ such that $|x_{n_k}-x| < \varepsilon, \forall n_k \geq M$.
\end{proof}

From the decision of subsequence we may ask: ``does a bounded sequence have a convergent subsequence?''. The answer is yes, before we show it, we need to define some specific limits.

\begin{definition}[Limsup/liminf]
    Let $\{x_n\}$ be a bounded sequence. If the limit exists, we can define:
    \begin{itemize}
        \item Limit superior: $\limsup \limits_{n \to \infty} x_n = \lim \limits_{n \to \infty} \left( \sup \{ x_k: k \geq n\}\right)$
        \item Limit inferior: $\liminf \limits_{n \to \infty} x_n = \lim \limits_{n \to \infty} \left( \inf \{ x_k: k \geq n\}\right)$
    \end{itemize}
\end{definition}

Now we proceed to show an interesting result: these limits always exist.

\begin{theorem}
    Let $\{x_n\}$ be a bounded sequence, and
    \begin{itemize}
        \item $a_n = \sup\{x_k: k \geq n\}$
        \item $b_n = \inf\{x_k: k \geq n\}$
    \end{itemize}
    Then, the following statements are true:
    \begin{enumerate}
        \item $\{a_n\}$ is monotone decreasing and bounded,
        \item $\{b_n\}$ is monotone increasing and bounded,
        \item $\liminf \limits_{n\to \infty} x_n \leq \limsup_{n \to \infty} x_n$
    \end{enumerate}
\end{theorem}

\begin{proof}
    Proving each of the results:
    \begin{enumerate}
        \item First, $\{x_k:k \geq n+1\} \subseteq \{x_k: k \geq n\}, \forall n \in N$, so $a_{n+1} = \sup \{x_k: k \geq n+1\} \leq \sup \{x_k: k \geq n\} = a_n$.
        \item Similarly, $b_{n+1} \geq b_n, \forall n \in \N$. Since $\{x_n\}$ is bounded, $\exists M \geq 0$ such that $-B \leq x_n \leq B, \forall n \in \N$. So, $-B \leq b_n \leq a_n \leq B$.
        \item By the previous result, $b_n \leq a_n, \forall n \in \N \Longrightarrow \liminf \limits_{n \to \infty} x_n = \lim \limits_{n \to \infty} b_n \leq \lim \limits_{n \to \infty} a_n = \limsup \limits_{n \to \infty} x_n$
    \end{enumerate}
\end{proof}

\paragraph{Example} Consider the series $\{x_n\}$ with $x_n=(-1)^n$. Calculate the limit superior and limit inferior.

\begin{proof}
    First, notice that $\{ (-1)^k: k \geq n\} = \{-1, 1\}, \forall n \in \N$. Hence, the supremum is always $1$ and the infimum is always $-1$. So,
    \begin{equation*}
        \limsup \limits_{n \to \infty} = 1
    \end{equation*}
    \begin{equation*}
        \liminf \limits_{n \to \infty} = -1
    \end{equation*}
\end{proof}
