\chapter{Sequences and series}

\section{The starting problem}

Basically a series is a sum of infinite terms. On the following example, some problems will appear as we try to manipulate the series as standard mathematical entities. Consider, for instance:

\begin{equation}
    \sum \limits_{n=1}^\infty \frac{(-1)^{n+1}}{n} = 1 - \frac{1}{2} + \frac{1}{3} - \frac{1}{4} + ...
\end{equation}

We can consider the partial sum, $s_n$, \emph{i.e.} the sum of the $n$ first terms of the series. In this case we would obtain: $s_1 = 1$, $s_2 = 1/2$, $s_3 = 5/6$,... and so on. Interestingly, the odd sums decrease ($s_1 > s_3 > s_5 > ...$), while the even sums increase ($s_2 < s_4 < s_6 < ...$). It gives the idea that $(s_n)$, the sequence of partial sums, converges to some number $S$. And we may feel tempted to write:

\begin{equation*}
    S = 1 - \frac{1}{2} + \frac{1}{3} - \frac{1}{4} + ...
\end{equation*}

However, the use of standard mathematical notation ($+, -, =$) for series can be misleading. Take the previous equation, multiply it for $1/2$ and add it to itself. We would get:

\begin{equation*}
    \frac{3}{2}S = 1 - \frac{1}{2} + \frac{1}{3} - \frac{1}{4} + ...
\end{equation*}

Which seems to be a contradiction to our initial claim. In a certain sense, addition in this infinite setting is not commutative.

Another example is the series:

\begin{equation}
    \sum \limits_{n=1}^\infty (-1)^n = -1 + 1 - 1 + 1 - 1 + ...
\end{equation}

Depending on how we group the terms we would find different results:

\begin{equation*}
    (-1 + 1) + (-1 + 1) + (-1 + 1) + ... = 0
\end{equation*}

On the other hand,

\begin{equation*}
    -1 + (1-1) + (1-1) + (1-1) + ... = 1
\end{equation*}

In order to deal with the tricks hidden in infinite series, we begin by discussing sequences.

\section{Sequences}

\subsection{Convergent sequences}

\begin{definition}[Sequence]
    A sequence is a function, $f$, whose domain is $\N$. In this way, $f: \N \to \R$. Hence, $f(n)$ is the $n$-th term of the sequence. Notation: usually, a sequence is presented in the form $(x_n)$, or $(x_n)_{n=1}^\infty$, or $ x_1, x_2, x_3, ...$.
\end{definition}

\begin{definition}[Convergence of a sequence]
    A sequence $(x_n)$ converges to $x$ if $\forall \varepsilon > 0, \exists N \in N$ such that $|x_n - x| < \varepsilon, \forall n \geq N$. There are a few different ways to denote convergence, such as $(x_n) \to x$, $\lim \limits_{n \to \infty} x_n = x$ or $x_n \to x$.
\end{definition}

The negation of the convergence of a sequence would be:

\begin{definition}
    A sequence  $( x_n)$ does not converge to $x$ if $\exists \varepsilon_0 > 0$ such that $\exists m \in \N$ such that $|x_n - x| \geq \varepsilon, \forall n \geq m$.
\end{definition}

\begin{eg}
    \begin{equation*}
        \lim \limits_{n \to \infty} \frac{1}{n^2 + 30n + 1} = 0
    \end{equation*}
    Solution:\\   
    We need to find $N \in \N$ such that
    \begin{equation*}
        \frac{1}{n^2 + 30n + 1} < \varepsilon, \forall n \geq N
    \end{equation*}
    But
    \begin{equation*}
        \frac{1}{n^2 + 30n + 1} \leq \frac{1}{n^2 + 30n} \leq \frac{1}{30n} \leq \frac{1}{n}
    \end{equation*}
    Hence, if $1/n < \varepsilon$ the initial inequality is immediately satisfied. \\
    Let $\varepsilon > 0$, set $N \in \N$ such that $1/N < \varepsilon$. Then, for all $n \geq N$:
    \begin{equation*}
        \left| \frac{1}{n^2+30n+1} - 0\right| = \frac{1}{n^2+30n+1} \leq  \frac{1}{30n} \leq \frac{1}{n} \leq \frac{1}{N} < \varepsilon
    \end{equation*}     
\end{eg}

\begin{definition}[Bounded sequences]
    A sequence $(x_n)$ is bounded if there exists a number $M > 0$ such that $|x_n| < M, \forall n \in \N$.
\end{definition}

\begin{theorem}
    If $( x_n)$ is convergent, then $( x_n)$ is bounded.
\end{theorem}

\begin{proof}
    Suppose $( x_n) \to x$, then $\exists N \in \N$ such that $\forall n \geq N, |x_n - x| < \varepsilon, \forall \varepsilon > 0$. Regardless if $x$ is positive or negative, we can write $|x_n| < |x| + \varepsilon$. Define $M = \max (||x_1|, |x_2|, ..., |x_{N-1}|, |x| + \varepsilon )$. Then, $|x_n| \leq M, \forall n \in \N$. 
\end{proof}

\begin{definition}
    A sequence $( x_n )$ is:
    \begin{enumerate}
        \item Monotone increasing, if $x_n \leq x_{n+1}, \forall n \in \N$,
        \item Monotone decreasing, if $x_n \geq x_{n+1}, \forall n \in \N$,
        \item If it is either monotone increasing or decreasing, then it is called monotone.
    \end{enumerate}
\end{definition}

\begin{theorem}
    A monotonic sequence is convergent if, and only if, it is bounded.
\end{theorem}

\begin{proof}
    Suppose $( x_n)$ is a monotonic increasing sequence. Then,
    \begin{enumerate}
        \item $( \Rightarrow )$ follows from the previous theorem.
        \item $( \Leftarrow )$. Suppose $ (x_n ) $ is bounded. Then, $ ( x_n: n \in \N) \subseteq \R$ is a bounded set. Let $x = \sup ( x_n: n \in \R)$. We claim 
        \begin{equation*}
            x = \lim \limits_{n\to \infty} x_n
        \end{equation*}
        Let $\varepsilon > 0$. Since $x - \varepsilon$ is not an upper bound for $( x_n: n \in \N)$, $\exists M_0 \in \N$ such that $x_n - \varepsilon < x_{M_0} < x$. Choose $M = M_0$, then $\forall n \geq M, x-\varepsilon < x_{M_0} < x_n \leq x + \varepsilon$, or $x-\varepsilon < x_M < x+\varepsilon$.
    \end{enumerate}
\end{proof}

\begin{theorem}[Algebraic limit theorem]
    Let $( a_n) \to a$ and $( b_n) to b$. Then,
    \begin{enumerate}
        \item $( ca_n) \to ca, \forall c \in \R$
        \item $( a_n + b_n) \to a + b$
        \item $( a_nb_n) \to ab $
        \item $(a_n/b_n) \to a/b$, given $b \ne 0$
    \end{enumerate}
\end{theorem}

\begin{proof}
    Let's take each item individually:
    \begin{enumerate}
        \item First, note $|ca_n - ca| = |c||a_n - a|$. Hence, for $\varepsilon > 0$ we have $|ca_n - ca| < \varepsilon \Leftrightarrow |a_n-a| < \varepsilon/|c|$. Since $( a_n ) \to a$ then $\exists N \in \N$ such that $|a_n - a| < \varepsilon / |c|$, so $|ca_n - ca| = |c||a_n-a| < |c|\varepsilon/|c|, \forall n \geq N$.
        \item From the triangle inequality, $|(a_n+b_n)-(a-b)| \leq |a_n-a|+|b_n-b|$. Set $N_1 \in \N$ such that $|a_n-a|< \varepsilon/2, \forall n \geq N_1$ with $\varepsilon > 0$. And set $N_2 \in \N$ such that $|b_n-b| < \varepsilon/2, \forall n \geq N_2$. Then, for $N = \max ( N_1, N_2)$ we obtain: $|(a_n+b_n)-(a+b)| \leq |a_n-a| + |b_n-b| < \varepsilon/2 + \varepsilon/2 = \varepsilon$.
        \item First, $|a_nb_n - ab| = |a_nb_n -ab_n + ab_n - ab| \leq |a_nb_n-ab_n| + |ab_n - ab| = |b_n||a_n-a| + |a||b_n-b|$. Take $N_1 \in \N$ such that $|b_n-b| < \varepsilon/(2|a|), \forall n \geq N_1$ with $\varepsilon > 0$. Since every convergent sequence is bounded, take $M > 0$ so that $|b_n| < M, \forall n \in \N$. Then, set $N_2 \in \N$ such that $|a_n-a| < \varepsilon/(2M), \forall n \geq N_2$. Finally, for $N = \max ( N_1, N_2)$ we obtain $|a_nb_n - ab| \leq |b_n||a_n-a| + |a||b_n-b| < M\varepsilon/(2M) + |a|\varepsilon/(2|a|) = \varepsilon$.
        \item $( a_n/b_n) \to a/b$ follows from the previous result by noting $((1/b_n)) \to 1/b$, provided $b \ne 0$. So,
        \begin{equation*}
            \left |
            \frac{1}{b_n} - \frac{1}{b}
            \right | = \frac{|b -b_n|}{|b||b_n|}
        \end{equation*}
        $|b_n-b|$ can be made arbitrarily small. On the other hand, considering $\varepsilon_0 = |b|/2$, define $N_1 \in \N$ such that $|b_n-b|<|b|/2, \forall n \geq N_1$, hence $|b_n| > |b|/2, \forall n \geq N_1$. Now, set $N_2 \in \N$ such that $|b_n-b| < \varepsilon|b|^2/2$. Taking $N = \max ( N_1, N_2)$ leads to
        \begin{equation*}
            \left|
                \frac{1}{b_n} - \frac{1}{b}
                \right| = 
                |b-b_n|\frac{1}{|b||b_n|} < \frac{\varepsilon |b|^2}{2} \frac{1}{|b|\frac{|b|}{2}} = \varepsilon, \forall n \geq N
        \end{equation*}
    \end{enumerate}
\end{proof}

\begin{theorem}[Order limit theorem]
    Assume $(a_n) \to a$ and $( b_n) \to b$, then:
    \begin{enumerate}
        \item If $a_n \geq 0, \forall n \in \N \Rightarrow a \geq 0$,
        \item If $a_n \leq b_n, \forall n \in \N \Rightarrow a \leq b$,
        \item If there exists $c \in \R$ such that $c \leq b_n, \forall n \in \N$, then $c \leq b$. Similarly, if $a_n \leq c, \forall n \in \N$ then $a \leq c$.
    \end{enumerate}
\end{theorem}

\begin{proof}
    Proof for each statement:
    \begin{enumerate}
        \item Assume $a < 0$. Consider $\varepsilon  = |a|$, since $( a_n) \to a$ we have $|a_n - a|< |a|, \forall n \geq N$. In particular, $|a_N - a|<|a|$ hence $a_N < 0$ which is a contradiction. Therefore, $a \geq 0$.
        \item From the algebraic theorem,  $( b_n - a_n) \to a-b$. Since $b_n-a_n \geq 0$ from the previous result, we get $b - a \geq 0$.
        \item Take $a_n = c, \forall n \in \N$. From the previous theorem, if $c \leq b_n$ then $b-c \geq 0$. Hence, $b \geq c$.
    \end{enumerate}
\end{proof}

\subsection{Operations involving convergent sequences}

In order to find out if a sequence converges, and to what value, there are a few tools at our disposal. We begin with the very popular Squeeze theorem, sometimes refered here as ST.

\begin{theorem}[Squeeze theorem]
    Let $(a_n), (b_n), (x_n)$ be sequences such that $\forall n \in \N$, $a_n \leq x_n \leq b_n$. Suppose $(a_n)$ and $(b_n)$ both converge and
    \begin{equation*}
        \lim \limits_{n \to \infty} a_n = x = \lim \limits_{n \to \infty} b_n
    \end{equation*}
    So, $(x_n) \to x$.
\end{theorem}

\begin{proof}
    Let $\varepsilon > 0$. Since $(a_n) \to x$, $\exists M_0 \in \N$ such that $\forall n \geq M_0, |a_n-x| < \varepsilon \Rightarrow x-\varepsilon < a_n$.\\
    Since $(b_n) \to x$, then $\exists M_1 \in \N$ such that $\forall n \geq M_1, |b_n-x|<\varepsilon \Rightarrow b_n < x + \varepsilon$. \\
    Choose $M = \max( M_0, M_1)$. Then, if $n \geq M, x-\varepsilon < a_n \leq x_n \leq b_n < x-\varepsilon \Rightarrow |x_n-x|>\varepsilon$.
\end{proof}

The limit of a function can be also expressed in another form, which can be pretty useful.

\begin{theorem}
    Another way to check if $(x_n) \to x$ would be
    \begin{equation*}
        \lim \limits_{n \to \infty} x_n = x \Longleftrightarrow \lim \limits_{n \to \infty} |x_n - x| = 0
    \end{equation*}
\end{theorem}

\begin{eg}
    Show that:
    \begin{equation*}
        \lim \limits_{n \to \infty} \frac{n^2}{n^2+n+1} = 1
    \end{equation*}    
    Solution:\\
    We have
    \begin{equation*}
        \left |
            \frac{n^2}{n^2+n+1} - 1
        \right | = 
        \left |
            \frac{-n-1}{n^2+n+1}
        \right | = 
        \frac{n+1}{n^2+n+1} \leq
        \frac{n+1}{n^2+n} = \frac{1}{n}
    \end{equation*}
    Thus,
    \begin{equation*}
        0 \leq 
        \left |
            \frac{n^2}{n^2+n+1} - 1
        \right |
        \leq
        \frac{1}{n}
        \Longrightarrow
        \lim \limits_{n \to \infty}
        \left |
            \frac{n^2}{n^2+n+1} - 1
        \right | = 0
    \end{equation*}
    by the Squeeze theorem.
\end{eg}

\subsection{Some special sequences}

\begin{theorem}
    If $(x_n)$ is a convergent sequence such that $\forall n \in \N, x_n \geq 0$, then $(\sqrt{x_n})$ is convergent and
    \begin{equation}
        \lim \limits_{n \to \infty} \sqrt{x_n} = \sqrt{\lim \limits_{n \to \infty} x_n}
    \end{equation}
\end{theorem}

\begin{proof}
    Let $x = \lim \limits_{n \to infty} x_n$, then:
    \begin{enumerate}
        \item If $x = 0$. Let $\varepsilon > 0$ then, since $(x_n) \to 0$, $\exists N_0 \in \N$ such that $x_n = |x_n-0| \varepsilon^2, \forall n \geq N_0$. Choose $N = N_0$, then $\sqrt{x_n}-\sqrt{0} = \sqrt{x_n} < \sqrt{\varepsilon^2} = \varepsilon$
        \item If $ x > 0$. Then, 
        
        \begin{align}
            |\sqrt{x_n} - \sqrt{x}| &= \left |
                \frac{\sqrt{x_n}-\sqrt{x}}{\sqrt{x_n}+\sqrt{x}}(\sqrt{x_n}+\sqrt{x})
            \right | \\
            &= \frac{1}{\sqrt{x_n} - \sqrt{x}}|x_n-x| \\
            &\leq \frac{1}{\sqrt{x}}|x_n-x|, \forall n \in \N
        \end{align}
        
        And,
        \begin{equation*}
            0 \leq |\sqrt{x_n}-\sqrt{x}| \leq \frac{1}{\sqrt{x}}|x_n-x|, \forall n \in \N
        \end{equation*}
        So, by the squeeze theorem, $\lim \limits_{n\to\infty} |\sqrt{x_n}-\sqrt{x}| = 0$
    \end{enumerate}
\end{proof}

\begin{theorem}
    If $(x_n)$ is convergent and $\lim \limits_{n\to\infty} x_n = x$, then $(|x_n|)$ is convergent and $\lim \limits_{n \to \infty} |x_n| = |x|$.
\end{theorem}

\begin{proof}
    Note that $|x| = \sqrt{x^2}, \forall x \in \R$. Then,
    \begin{equation*}
        \lim \limits_{n\to \infty} |x_n| = \lim \limits_{n\to \infty} \sqrt{x_n^2} = \sqrt{x^2} = x
    \end{equation*}
\end{proof}

\begin{theorem}
    If $x \in (0,1)$ then $\lim \limits_{n \to \infty} c^n = 0$. If $c > 1$ then $(c^n)$ is unbounded.
\end{theorem}

\begin{proof}
    For each case:
    \begin{enumerate}
        \item If $0 < c < 1$. Note that $0 < c^{n+1} < c^n < 1, \forall n \in \N$. This can be shown by induction:
        \begin{itemize}
            \item Base case, consider $0 < c^2 < c < 1$, since $0 < c < 1$.
            \item Inductive case, consider $0 < c^{m+1} < c^m < 1$ to be true. Multiplying the former inequality by $c$ we obtain $0 < c^{m+2} < c^{m+1}$.
        \end{itemize}
        Thus, $(c^n)$ is monotone decreasing sequence and is bounded below, which implies $(c^n)$ is convergent. Set $L = \lim \limits_{n \to \infty} c^n$. Take $\varepsilon > 0$, then $\exists N \in \N$ such that
        
        \begin{align*}
            (1-c)|L| &= |L-cL| = |L - c^{M+1} + c^{M+1} - cL| \\
            &\leq |L-c^{M+1}| + c|c^M-L| \\
            &< (1-c)\frac{\varepsilon}{2} + c(1-c)\frac{\varepsilon}{2} \\
            &< (1-c) \varepsilon, \forall n \geq N
        \end{align*}
        
        Hence, $|L| < \varepsilon, \forall \varepsilon>0 \longrightarrow L = 0$.
        \item For $c > 1$. Note that $\forall B \geq 0, \exists n \in \N$ such that $c^n > B$. For $n \in \N$ such that $n > B/(c-1)$ then $c^n = (1+(1-c))^n \geq 1 + n(c-1) \geq n(c-1) > B$. Hence, $(c^n)$ is unbounded $\forall c > 1$ and the sequence does not converge.
    \end{enumerate}
\end{proof}

\begin{theorem}
    If $p>0$ then $\lim \limits_{n \to \infty}n^{-p} = 0$
\end{theorem}

\begin{proof}
    Let $\varepsilon > 0$. Take $N > (1/\varepsilon)^{1/p}$, then
    \begin{equation*}
        \left |
        \frac{1}{n^p} - 0
        \right | = 
        \frac{1}{|n^p|} \leq
        \frac{1}{N^p} <
        \varepsilon
    \end{equation*}
\end{proof}

\begin{theorem}
    If $p>0$ then $\lim \limits_{n \to \infty} p^{1/n}=1$.
\end{theorem}

\subsection{Subsequences and Bolzano–Weierstrass theorem}

\begin{definition}[Subsequence]
    Let $(x_n)$ be a sequence of real numbers, and $(n_k)$ be a strictly increasing sequence of natural numbers. Then, $(x_{n_k})_{k=1}^\infty$ is called a subsequence of $(x_n)$.
\end{definition}

\begin{theorem}
    If $(x_n)$ converges to $x$ then any subsequence of $(x_n)$ will converge to $x$.
\end{theorem}

\begin{proof}
    Suppose $(x_n) \to x$. Let $\varepsilon > 0$, then $\exists M_0 \in \N$ such that $\forall n \geq M_0, |x_n-x| < \varepsilon$. Choose, $M=M_0$. If $k \geq M$, then $n_k \geq k \geq M = M_0$, hence for $\varepsilon>0, \exists M \in \N$ such that $|x_{n_k}-x| < \varepsilon, \forall n_k \geq M$.
\end{proof}

From the decision of subsequence we may ask: ``does a bounded sequence have a convergent subsequence?''. The answer is yes, before we show it, we need to define some specific limits.

\begin{definition}[Limsup/liminf]
    Let $(x_n)$ be a bounded sequence. If the limit exists, we can define:
    \begin{itemize}
        \item Limit superior: $\limsup \limits_{n \to \infty} x_n = \lim \limits_{n \to \infty} \left( \sup \{ x_k: k \geq n\}\right)$
        \item Limit inferior: $\liminf \limits_{n \to \infty} x_n = \lim \limits_{n \to \infty} \left( \inf \{ x_k: k \geq n\}\right)$
    \end{itemize}
\end{definition}

Now we proceed to show an interesting result: these limits always exist.

\begin{theorem}
    Let $(x_n)$ be a bounded sequence, and
    \begin{itemize}
        \item $a_n = \sup\{x_k: k \geq n\}$
        \item $b_n = \inf\{x_k: k \geq n\}$
    \end{itemize}
    Then, the following statements are true:
    \begin{enumerate}
        \item $(a_n)$ is monotone decreasing and bounded,
        \item $(b_n)$ is monotone increasing and bounded,
        \item $\liminf \limits_{n\to \infty} x_n \leq \limsup_{n \to \infty} x_n$
    \end{enumerate}
\end{theorem}

\begin{proof}
    Proving each of the results:
    \begin{enumerate}
        \item First, $\{x_k:k \geq n+1\} \subseteq \{x_k: k \geq n\}, \forall n \in N$, so $a_{n+1} = \sup \{x_k: k \geq n+1\} \leq \sup \{x_k: k \geq n\} = a_n$.
        \item Similarly, $b_{n+1} \geq b_n, \forall n \in \N$. Since $(x_n)$ is bounded, $\exists M \geq 0$ such that $-B \leq x_n \leq B, \forall n \in \N$. So, $-B \leq b_n \leq a_n \leq B$.
        \item By the previous result, $b_n \leq a_n, \forall n \in \N \Longrightarrow \liminf \limits_{n \to \infty} x_n = \lim \limits_{n \to \infty} b_n \leq \lim \limits_{n \to \infty} a_n = \limsup \limits_{n \to \infty} x_n$
    \end{enumerate}
\end{proof}

\begin{eg}
    Consider the sequence $(x_n)$ with $x_n=(-1)^n$. Calculate the limit superior and limit inferior.\\
    \textbf{Solution}: \\
    First, notice that $\{ (-1)^k: k \geq n\} = \{-1, 1\}, \forall n \in \N$. Hence, the supremum is always $1$ and the infimum is always $-1$. So,
    \begin{equation*}
        \limsup \limits_{n \to \infty} = 1
    \end{equation*}
    \begin{equation*}
        \liminf \limits_{n \to \infty} = -1
    \end{equation*}
\end{eg}


\begin{theorem}[Bolzano-Weierstrass]
    Every bounbed sequence has a convergent subsequence.
\end{theorem}

\begin{proof}
    This result follows from the previous theorem.
\end{proof}

\begin{theorem}
    Let $(x_n)$ be a bounded subsequence. Then, $(x_n)$ converges if, and only if:
    \begin{equation*}
        \liminf \limits_{n \to \infty} x_n = \limsup \limits_{n \to \infty} x_n
    \end{equation*}
\end{theorem}

\begin{proof}
    Proving each direction separately:
    \begin{itemize}
        \item $(\Longrightarrow)$: let $x = \lim \limits_{n \to \infty} x_n$ then every subsequence converges to $x$, so $\liminf \limits_{n \to \infty} x_n = x$ and $\limsup \limits_{n \to \infty} x_n = x$. Hence, $\liminf \limits_{n \to \infty} x_n = \limsup \limits_{n \to \infty} x_n$.
        \item $(\Longleftarrow)$: suppose $\liminf \limits_{n \to \infty} x_n= \limsup \limits_{n \to \infty} x_n$. Then, $\inf \{ x_k: k \geq n\} \leq x_k \leq \sup \{x_k: k \geq n\}, \forall n \in \N$. By the squeeze theorem we obtain: $\lim \limits_{n \to \infty} x_n  = \liminf \limits_{n \to \infty} x_n = \limsup \limits_{n \to \infty} x_n$.
    \end{itemize}
\end{proof}

\subsection{Cauchy sequences}

\begin{definition}[Cauchy sequence]
    A sequence $(x_n)$ is Cauchy if $\forall \varepsilon > 0, \exists N \in \N$ such that $|x_n-x_k| < \varepsilon, \forall n,k \geq N$.
\end{definition}

\begin{eg}
    Show $x_n = 1/n$ is Cauchy. \\
    \textbf{Solution}: \\
    Let $\varepsilon > 0$, take $N \in \N$ such that $1/N < \varepsilon/2$. Then,
    \begin{equation*}
        \left |
            \frac{1}{n} - \frac{1}{k}
        \right | \leq
        \frac{1}{n} + \frac{1}{k} \leq
        \frac{2}{N} < \varepsilon
    \end{equation*} 
\end{eg}

\begin{theorem}
    If $(x_n)$ is Cauchy, then $(x_n)$ is bounded.
\end{theorem}

\begin{proof}
    From the definition of a Cauchy sequence, take $\varepsilon=1$ without loss of generality, then $\exists N \in \N$ such that $|x_n-x_k| < 1, \forall n, k \geq N$. So, $|x_n| \leq |x_n - x_N| + |x_N| < |x_M| + 1$. Let $M = |x_1| + |x_1| + ... + |x_M| + 1$, so $|x_n| \leq M, \forall n \in \N$.
\end{proof}

\begin{theorem}
    IF $(x_n)$ is Cauchy and a subsequence $(x_{n_k})$ converges, then $(x_n)$ converges.
\end{theorem}

\begin{proof}
    Suppose $(x_{n_k})$ is a subsequence of $(x_n)$ such that $\lim \limits_{n \to \infty} x_n = x$. Let $\varepsilon > 0$, then $\exists N_1 \in \N$ such that $|x_{n_k} - x| < \varepsilon/2, \forall k \geq N_1$. Since $(x_n)$ is Cauchy, $\exists N_2 \in \N$ such that $|x_n-x_m| < \varepsilon/2, \forall n, m \geq M_1$.\\
    Take $N = \max \{N_1, N_2\}$. Then, $|x_n-x| \leq |x_n-x_{n_N}| + |x_{n_M} - x| < \varepsilon/2  + \varepsilon/2 = \varepsilon$.
\end{proof}

\begin{theorem}
    A sequence $(x_n)$ is Cauchy if, and only if, it is convergent.
\end{theorem}

\begin{proof}
    Proving each direction of the theorem:
    \begin{itemize}
        \item $(\Longrightarrow)$: if $(x_n)$ is Cauchy then it is bounded. So, by the Bolzano-Weierstrass theorem, there exists a convergent subsequence. From the previous theorem, if a sequence is Cauchy and has a convergent subsequence, then it is convergent. So, $(x_n)$ is convergent.
        \item $(\Longleftarrow)$: Suppose $(x_n)$ is convergent and $(x_n) \to x$. Let $\varepsilon > 0$, then $\exists N_0 \in \N$ such that $|x_n - x| < \varepsilon/2, \forall n \geq N_0$. Choose $N = N_0$, then $|x_n-x_k| \leq |x_n-x| + |x_k-x| < $
    \end{itemize}
\end{proof}

\section{Series}

\subsection{Convergent series}

As pointed out by David Bressoud in `` A radical approach to real analysis'', the infinite summation is in itself an oxymoron. That is, the sum is the process of adding up, or reaching the totality. On the other hand, infinite means never-ending. Weird things can happen as we deal with series. The formal treatment of real analysis aims to safeguard us against danger. 

\begin{definition}[Series convergence]
    Given the sequence, $(x_n)$, the series associated with it is the summation of its terms, denoted as $\sum \limits_{n=1}^\infty x_n = \sum x_n$. The series converges the sequence of partial sums
    \begin{equation}
        \left (
            s_m = \sum \limits_{n=1}^m x_n
        \right ) _{m=1}^\infty
    \end{equation}
    converges. If $\lim \limits_{m \to \infty} s_m = s$ we write $\sum x_n = s$ and treat $\sum x_n$ as a number.
\end{definition}

\begin{eg}
    Prove that the series $\sum _{n=1}^\infty 1/(n(n+1))$ converges. \\
    \textbf{Solution}: \\
    First, note that
    \begin{align*}
        s_m = \sum \limits_{n=1}^\infty \frac{1}{n(n+1)} &= \sum \limits_{n=1}^\infty \frac{1}{n} - \frac{1}{n+1} \\
        &= \left(
            1 + \frac{1}{2} + \frac{1}{3} + ... + \frac{1}{m}
        \right) - 
        \left(
            \frac{1}{2} + \frac{1}{3} + ... + \frac{1}{m+1}
        \right) \\
        &= 1 -\frac{1}{m+1}
    \end{align*}
    Thus, $s_m = 1 - \frac{1}{m+1} \to 1$ and the series converges.
\end{eg}

\begin{theorem}[Geometric series convergence]
    If $|x| < 1$ then $\sum _{n=0}^\infty r^n$ converges and
    \begin{equation}
        \sum \limits_{n=0}^\infty r^n = \frac{1}{1-x}
    \end{equation}
\end{theorem}

\begin{proof}
    First, note that
    \begin{equation*}
        s_m = \sum \limits_{n=0}^m x^n = \frac{1-x^{m+1}}{1-x}, \forall m \in \N
    \end{equation*}
    which can be proven by induction (not the point here). Since $|x| < 1$ we have $\lim_{m\to \infty} |x|^{m+1} = 0$ and
    \begin{equation*}
        \lim \limits_{n \to \infty} s_m = \frac{1-0}{1-x} = \frac{1}{1-x}
    \end{equation*}
\end{proof}

\begin{theorem}
    Consider the sequence $(x_n)$ and let $N \in \N$. Then, $\sum_{n=1}^\infty x_n$ converges if and only if $\sum_{n=N}^\infty x_n$ converges.
\end{theorem}

\begin{proof}
    The partial sums satisfy
    \begin{equation*}
        \sum \limits_{n=1}^N x_n = \sum \limits_{n=1}^M x_n + \sum \limits_{n=M}^N x_n, \forall  N \in \N \textnormal{ and } 1 \leq M \leq N
    \end{equation*}
\end{proof}

\begin{definition}
    The series $\sum x_n$ is Cauchy if the sequence of partial sums $(s_m = \sum_{n=1}^m x_n)$ is Cauchy.
\end{definition}

\begin{theorem}
    The series $\sum x_n$ is Cauchy if, and only if, $\sum x_n$ is convergent.
\end{theorem}

\begin{proof}
    This follows from the previous theorem which states that a sequence is Cauchy if, and only if, it is convergent.
\end{proof}

\begin{theorem}
    The series $\sum x_n$ is Cauchy if, and only if, $\forall \varepsilon > 0, \exists N \in \N$ such that $\forall m \geq N$ and $l \geq m$,
    \begin{equation*}
        \left |
            \sum \limits_{n=m+1}^l x_n 
        \right |
        < \varepsilon
    \end{equation*}
\end{theorem}

\begin{proof}
    $(\Longrightarrow)$ If $\sum x_n$ is Cauchy, let $\varepsilon > 0$ then $\exists N_0 \in \N$ such that $|s_m-s_l| < \varepsilon, \forall m,l \geq N_0$, where $s_m$ is the partial sum of the first $m$ terms. Hence, for $N = N_0$ if $m \geq N$ and $l > m$ we obtain:
    \begin{equation*}
        |s_l - s_m| 
        =\left |
            \sum \limits_{n=m+1}^l x_n 
        \right |
        < \varepsilon
    \end{equation*}
    $(\Longleftarrow)$ To-do
\end{proof}

\begin{theorem}
    If $\sum x_n$ converges, then $(x_n) \to 0$.
\end{theorem}

\begin{proof}
    If $\sum x_n$ converges then it is Cauchy, hence for $\varepsilon > 0, \exists N_0$ such that $\forall l > m \geq N_0$, 
    \begin{equation*}
        \left |
            \sum \limits_{n=m+1}^l x_n
        \right | < \varepsilon
    \end{equation*}
    For $N  = N_0 + 1$, then $m \geq N$ implies $m-1 \geq N_0$. For $l = m$:
    \begin{equation*}
        |s_m| = \left |
            \sum \limits_{n=m}^m x_n
        \right |
        < \varepsilon
    \end{equation*}
    And since $\varepsilon$ can be made arbitrarily small $(x_n) \to 0$.
\end{proof}

\begin{theorem}
    If $|x| \geq 1$ then $\sum_{n=0}^\infty x^n$ diverges.
\end{theorem}

\begin{proof}
    If $|x| \geq 1$ then $\lim_{n \to \infty} x_n \neq 0$ and $\sum x^n$ diverges from the previous theorem.
\end{proof}

\begin{corollary}
    The series $\sum _{n=0}^\infty \alpha \cdot x^n$ converges if, and only if, $|x| < 1$.
\end{corollary}

First, let's revisit one of the last results, namely that for a convergent series $\sum x_n$ we have $(x_n) \to x$. Is the converse true? That is, if $(x_n) \to 0$ can we conclude $\sum x_n$ converges? The answer is no, we will show it with a counter-example.

\begin{theorem}[Divergence of the harmonic series]
    The harmonic series $\sum_{n=1}^\infty 1/n$ does not converge.
\end{theorem}

\begin{proof}
    We will show there exists a subsequence $s_m = \sum_{n=1}^m 1/n$ which is unbounded, hence the series diverges. Consider:
    \begin{align*}
        s_{2^l} &= 1 + \left( \frac{1}{2}\right) + \left( \frac{1}{3} + \frac{1}{4}\right) +
        \left( \frac{1}{5} + ... + \frac{1}{8}\right) + ... +
        \left( \frac{1}{2^{l-1}+1} + \frac{1}{2^l}\right) \\
        &= 1 + \sum \limits_{\lambda = 1}^l \sum \limits_{n=2^{\lambda - 1} + 1}^{2^\lambda} \frac{1}{n}  \\
        &\geq 1 + \sum \limits_{\lambda = 1}^l \sum \limits_{n=2^{\lambda - 1} + 1}^{2^\lambda} \frac{1}{2^\lambda}  \\
        &= 1 + \sum \limits_{\lambda = 1}^l \frac{1}{2^\lambda}(2^\lambda - (2^{\lambda-1}+ 1) + 1) \\
        &= 1 + \sum \limits_{\lambda = 1}^l \frac{2^{\lambda-1}}{2^\lambda} \\
        &= 1 + \frac{l}{2}
    \end{align*}
    Hence, $(s_{2^l})_{l=1}^\infty$ is unbounded and by consequence $( s_{2^l})$ does not converge.
\end{proof}

\subsection{Properties of series}

\begin{theorem}[Algebraic limit theorem for series]
    Consider two convergent series: $\sum_{n=1}^\infty x_n = X$ and $\sum_{n=1}^\infty y_n = Y$. Then:
    \begin{enumerate}
        \item $\sum_{n=1}^\infty cx_n = cX, \forall c \in \R$,
        \item $\sum_{n=1}^\infty (x_n + y_n) = X + Y$
    \end{enumerate}
\end{theorem}

\begin{proof}
    Proving each of the statements:
    \begin{enumerate}
        \item First, note that the sequence of partial sums takes the form $t_m = c x_1 + c x_2 + ... + c x_m$, which is equivalent to $t_m = c s_m$ where $s_m = x_1 + x_2 + ... + x_m$. By the algebraic limit theorem, if the sequence $(s_m) \to x$ then $(t_m) = (c s_m) \to c X$.
        \item Equivalently, we have $s_m = x_1 + x_2 + ... + x_3$ and $t_m = y_1 + y_2 + ... + y_m$. And, $u_m = (x_1+y_1) + (x_2+y_2) + ... + (x_m+y_m) $, which is equivalent to $u_m = s_m + t_m$. Since $(s_m) \to X$ and $(t_m) \to Y$, then by the algebraic limit theorem $(u_m) = (s_m + t_m) \to X + Y$.
    \end{enumerate}
\end{proof}

\begin{theorem}
    If $x_n \geq 0, \forall n \in \N$ then $\sum x_n$ converges if, and only if, the sequence of partial sums, $(s_m)$, is bounded.
\end{theorem}

\begin{proof}
    If $x_n \geq 0, \forall n \in \N$ then
    \begin{equation*}
        s_{m+1} = \sum \limits_{n=1}^{m+1} x_n = \sum \limits_{n=1}^m + x_{m+1} = s_m + x_{m+1} \geq s_m
    \end{equation*}
    Which implies $(s_m)$ is a monotone increasing sequence, which converges if, and only if, it is bounded.
\end{proof}

\begin{definition}[Absolute convergence]
    The series $\sum x_n$ converges absolutely if $\sum |x_n|$ converges
\end{definition}

\begin{theorem}
    If $\sum x_n$ converges absolutely then $\sum x_n$ converges.
\end{theorem}

\begin{proof}
    If $\sum |x_n|$ converges then from the previous theorem $(s_m)$ is bounded, since $|x_n| \geq 0, \forall n \in \N$ and $s_m = \sum_{n=1}^m |x_n|$. From the previous result $s_{m+1} \geq s_m$. Since $\sum |x_n|$ converges by the initial hypothesis that means $\sum |x_n|$ is Cauchy. So, $\forall \varepsilon > 0, \exists N_0 \in \N$ such that:
    \begin{equation*}
        \sum \limits_{n=m+1}^l |x_n| < \varepsilon, \forall l > m \geq N_0
    \end{equation*}
    Take $N = N_0$. Then, 
    \begin{equation*}
        \left |
            \sum \limits_{n=m+1}^l x_n
        \right | \leq
        \sum \limits_{n=m+1}^l |x_n|
        < \varepsilon
    \end{equation*}
    Hence, $\sum x_n$ is Cauchy and converges.
\end{proof}

From the previous theorem, we may be tempted to ask `` If a series converges, does it mean it also converges absolutely?''. We will show this is not the case by a counter-example:

\begin{eg}
    Consider the series
    \begin{equation*}
        \sum \limits_{n=1}^\infty \frac{(-1)^n}{n}
    \end{equation*}
    This series converges, as we will see in a while. Now, show it does not converge absolutely.\\
    \textbf{Solution:}\\
    Note that the sum of absolute values is exactly the harmonic series, which we have already seen does not converge. 
\end{eg}

\subsection{Convergence tests}

\begin{theorem}[Comparison test]
    Suppose $0 \leq x_n \leq y_n, \forall n \in \N$, then:
    \begin{enumerate}
        \item if $\sum y_n$ converges, then $\sum x_n$ converges,
        \item if $\sum x_n$ diverges, then $\sum y_n$ diverges.
    \end{enumerate}
\end{theorem}

\begin{proof}
    Proving each statement:
    \begin{enumerate}
        \item If $\sum y_n$ converges, then the sequence of partial, $(\sum_{n=1}^m y_n)$ , sums is bounded. So, $\exists M \geq 0$ such that:
        \begin{equation}
            \sum \limits_{n=1}^m y_n \leq B, \forall m \in \N
        \end{equation}
        Hence, $0 \leq x_n \leq y_n, \forall n \in \N \Longrightarrow 0 \leq \sum_{n=1}^m x_n \leq \sum_{n=1}^m y_n \leq B, \forall m \in \N$, which implies $(\sum_{n=1}^m x_n)$ is bounded and therefore it converges.
        \item if $\sum x_n$ diverges, then $(\sum_{n=1}^m x_n)$ is unbounded. Let $M \geq 0$, then:
        \begin{equation}
            \sum \limits_{n=1}^m x_n \geq M, \forall m \in \N
        \end{equation}
        Therefore, $\sum_{n=1}^m y_n \geq \sum_{n=1}^m x_n \geq M$. which implies $(\sum_{n=1}^m y_n)$ is unbounded and therefore diverges.
    \end{enumerate}
\end{proof}

\begin{theorem}[Ratio test]
    Consider the series $\sum x_n$ and $x_n \neq 0, \forall n \in \N$. Suppose
    \begin{equation}
        L = \lim \limits_{n \to \infty} \frac{|x_{n+1}|}{|x_n|}
    \end{equation}
    exists. Then,
    \begin{enumerate}
        \item if $L < 1$ then $\sum x_n$ converges absolutely
        \item if $L > 1$ then $\sum x_n$ diverges
        \item if $L = 1$ no assertion can be made
    \end{enumerate}
\end{theorem}

\begin{proof}
    Proving the first two statements:
    \begin{enumerate}
        \item For $L < 1$, take $\alpha \in (L, 1)$. Then, $\exists N_0 \in \N$ such that $|x_{n+1}|/|x_n| < \alpha, \forall n \geq N_0$. Equivalently $|x_{n+1}| \leq \alpha |x_n|, \forall n \geq N_0$. Which leads to $|x_n| \leq \alpha |x_{n+1}| \leq \alpha^2 |x_{n+2}| \leq ... \leq \alpha^{n-N_0} |x_{n+1}|^{n-N_0}, \forall n \geq N_0$. Now, for $m \in \N$:
        \begin{align*}
            \sum \limits_{n=1}^m |x_n| &= \sum \limits_{n=1}^{N_0-1}|x_n| + \sum \limits_{n=N_0}^{m}|x_n| \\
            &\leq \sum \limits_{n=1}^{N_0-1}|x_n| + |x_{N_0}| \sum \limits_{n=N_0}^{m} \alpha^{n-N_0} \\
            &\leq \sum \limits_{n=1}^{N_0-1}|x_n| + |x_{N_0}| \sum \limits_{l=0}^{\infty} \alpha^l \\
            &= \sum \limits_{n=1}^{N_0-1}|x_n| + \frac{|x_{N_0}|}{1-\alpha}
        \end{align*}
        Therefore, $( \sum_{n=1}^m |x_n|)_{m=1}^\infty$ is bounded and $\sum |x_n|$ converges.
        \item For $L > 1$, take $\alpha \in (1, L)$. Then, $\exists N_0 \in \N$ such that $|x_{n+1}|/|x_n| \geq \alpha > 1, \forall n \geq N_0$. Which means $|x_{n+1}| \geq |x_n|, \forall n \geq N_0$. Hence, $\lim_{n \to \infty} |x_n| \neq 0$ and $\sum x_n$ diverges.
    \end{enumerate}
\end{proof}

\begin{eg}
    Consider the series
    \begin{equation*}
        \sum \limits_{n=1}^\infty \frac{(-1)^n}{n^2+1}
    \end{equation*}
    Use the ratio test to verify it converges absolutely. \\
    \textbf{Solution}: \\
    We begin by noticing:
    \begin{equation*}
        \left |
            \frac{(-1)^n}{n^2+1}
        \right | \leq
        \frac{1}{n^2+1} <
        \frac{1}{n^2}
    \end{equation*}
    So, for the limit:
    \begin{equation*}
        \lim \limits_{n \to \infty} \left |
            \frac{
                \frac{(-1)^{n+1}}{(n+1)^2+1}
            }{
                \frac{(-1)^{n}}{(n)^2+1}
            }
        \right | <
        \lim \limits_{n \to \infty} \frac{n^2}{(n+1)^2} = 1
    \end{equation*}
    Since the limit is less then $1$, the series converges absolutely by the ration test.
\end{eg}

\begin{theorem}[Root test]
    Consider the series $\sum x_n$ and suppose the limit
    \begin{equation*}
        L = \lim \limits_{n\to \infty} |x_n|^{1/n}
    \end{equation*}
    exists. Then,
    \begin{enumerate}
        \item if $L < 1$ the series converges absolutely,
        \item if $L > 1$ the series diverges,
        \item if $L = 1$ no assertion can be made.
    \end{enumerate}
\end{theorem}

\begin{proof}
    Proving the first two assertions:
    \begin{enumerate}
        \item Take $r \in (L, 1)$. Since $(|x_n|^{1/n} \to L), \exists N \in \N$ such that $|x_n|^{1/n} < r, \forall n \geq N$ which is equivalent to $|x_n| \leq r^n, \forall n \geq N$. So,
        \begin{align*}
            \sum \limits_{n=1}^n|x_n| &= \sum \limits_{n=1}^{N-1}|x_n| + \sum \limits_{n=N}^{m}|x_n| \\
            &\leq \sum \limits_{n=1}^{N-1}|x_n| + \sum \limits_{n=N}^{m}r^n \\
            &\leq \sum \limits_{n=1}^{N-1}|x_n| + \sum \limits_{n=1}^{\infty}r^n \\
            &= \sum \limits_{n=1}^{N-1}|x_n| + \frac{1}{1-r}
        \end{align*}
        Thus, the sequence of partial sums of absolute values is bounded and the series converges absolutely.
        \item Since $(|x_n|^{1/n}) \to L > 1, \exists N \in \N$ such that $|x_n|^{1/n} > 1, \forall n \geq N$, which is equivalent to $|x_n| > 1, \forall n \geq N$ and $\lim_{n \to \infty} x_n \neq 0$ and the series diverges.
    \end{enumerate}
\end{proof}

\begin{theorem}[Alternating series test]
    Consider the sequence $(x_n)$ to be monotone decreasing, with $x_n \to x$. Then $\sum (-1)^n x_n$ converges.
\end{theorem}

\begin{proof}
    Considering $(x_n)$ is monotone decreasing, then for the partial sums:
    \begin{align*}
        s_{2k} &= \sum \limits_{n=1}^{2k} (-1)^n x_n \\
        &= (x_2-x_1) + (x_4-x_3) + ... + (x_{2k} - x_{2k-1}) \\
        &\geq (x_2-x_1) + (x_4-x_3) + ... + (x_{2k} - x_{2k-1}) + (x_{2k+2} - x_{2k+1}) \\
        &= s_{2(k+1)}
    \end{align*}
    Hence, $(s_{2k})$ is monotone decreasing. And, $s_{2k} = -x_1 + (x_2-x_3) + (x_4-x_5) + ... + (x_{2k-2}-x_{2k-1}) + x_{2k} \geq -x_1$. So $(s_{2k})$ is monotone decreasing and bounded below. Therefore it converges.  \\
    Take $s = \lim_{k\to \infty} s_{2k}$ and $\varepsilon > 0$. Since $s_{2k} \to s, \exists N_1 \in \N$ such that $|s_{2k} - s| < \varepsilon/2, \forall k \geq N_1$. On the other hand, since $x_n \to 0, \exists N_2 \in \N$ such that $|x_n| < \varepsilon/2, \forall n \geq N_2$. Choose $N = \max(2N_1 + 1, N_2)$, then $|s_m-s| = |s_{2m/2}-s| < \varepsilon/2 < \varepsilon, \forall m \in \geq N$, with $m$ even and $m/2 \geq N_1 + 1/2 \geq M_0$. \\
    If $m$ is even, take $k=(m-1)/2$ so $m = 2k + 1$. Then, $|s_m-s| = |s_{m-1} + x_m - s| \leq |s_{2k} - s + x_m| \leq |s_{2k} - s| + |x_m| < \varepsilon/2 + \varepsilon/2 = \varepsilon$.
    Thus, $s_m \to s$ and $\sum (-1)^n x_n$ converges.
\end{proof}
\vspace{1em}

\begin{remark}
    From the previous theorem, it is clear that $\sum (-1)^n/n$ converges since $(1/n)$ is monotone decreasing, however as it was shown before it does not converge absolutely as the sum of the absolute terms of the series is exactly the harmonic series, which does not converge.
\end{remark}

\begin{theorem}
    Suppose $\sum x_n$ converges absolutely to $X$. Consider $f: \N \to \N$ a bijective function, then $\sum x_{f(n)}$ converges absolutely and $\sum x_{f(n)}$ converges to $X$. In other words ``If a series converges absolutely, then any rearrangement of its terms also converges to the same limit''.
\end{theorem}

\begin{proof}
    To show $\sum |x_{f(n)}|$ converges it suffices to show $\sum_{n=1}^m |x_{f(n)}|$ is bounded. Since $\sum x_n$ converges, then it is also bounded hence $\exists M \in \N$ such that
    \begin{equation*}
        \sum \limits_{n=1}^l |x_n| \leq M, \forall l \in \N
    \end{equation*}
    Let $m \in \N$, then $f((1, 2, ..., m))$ is a finite subset of $\N$, so $\exists l \in \N$ such that $f((1, 2, ..., m)) \subseteq (1, 2, ..., l)$. And,
    \begin{equation*}
        \sum \limits_{n=1}^n |x_{f(n)}| = \sum \limits_{n \in f((1, 2, ..., m))} |x_n| \leq \sum \limits_{n=1}^l |x_n| \leq M
    \end{equation*}
    So, $\sum |x_{f(n)}|$ converges. Let $\varepsilon  > 0$ then $\exists N_0 \in \N$ such that:
    \begin{equation*}
        \left |
            \sum \limits_{n=1}^m x_n - X
        \right | < \varepsilon/2, \forall m \geq N_0
    \end{equation*}
    Since $\sum |x_n|$ converges, $\exists N_1 \in \N$ such that:
    \begin{equation*}
        \sum \limits_{n=m+1}^m |x_n| < \varepsilon/2, \textnormal{ with } l > m > N_1
    \end{equation*}
    Take $N_2 = \max(N_0, N_1)$, then:
    \begin{equation*}
        \left |
            \sum \limits_{n=1}^m x_n - X
        \right | < \varepsilon/2
        \textnormal{ and }
        \sum \limits_{n=m+1}^m |x_n| < \varepsilon/2
        \textnormal{ and }
        l > m \geq N_2
    \end{equation*}
    Since $f^{-1}((1, ..., N_2))$ is finite, set $N \in \N$ such that $(1, ..., N_2) \subseteq (1, ..., N)$, then:
    \begin{align*}
        \left |
            \sum \limits_{n'=1}^{m'} x_{f(n')} - X
        \right | &= 
        \left |
            \sum \limits_{n \in f(( 1, ..., m'))} x_n - X
        \right | \\
        &= \left |
            \sum \limits_{n=1}^N x_n - X
            + \sum \limits_{n \in f((1, ..., m')) \setminus (1, ..., N)} x_n
        \right | \\
        &\leq
        \left |
            \sum \limits_{n=1}^N x_n - X
        \right | + \sum \limits_{n=N+1}^{\max (f((1, ..., m'))} |x_n|  \\
        &\leq 
        \left |
            \sum \limits_{n=1}^N x_n - X
        \right | + \sum \limits_{n=N+1}^{l} |x_n|  \\
        &< \varepsilon/2 + \varepsilon/2 = \varepsilon
    \end{align*} 
\end{proof}

