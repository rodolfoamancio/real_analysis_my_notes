\chapter{The derivative}

Before we begin dealing with derivative, we must build some of the tools needed to define it and evaluate its existence. We begin by evaluating limits on a function, which lead us to the cornerstone concept of continuity. 

\section{Continuity}

In order to expand our concepts of limits from sequences to functions on real numbers, we begin by defining the points on the real line we will be able to evaluate the future concepts.

\subsection{Limits of function}

\begin{definition}[Cluster point]
    Let $S \subseteq \R$. Then, $x \in \R$ is a cluster point of $S$ if $\forall \delta > 0, (x-\delta, x+ \delta) \cap S \setminus\{x\} \neq \varnothing$
\end{definition}

This notion can be further clarified by a few examples:
\paragraph{Examples}
\begin{enumerate}
    \item $S = \{1/n: n \in \N\}$: $0$ is a cluster point of $S$, since $1/n$ can be made arbitrary small, so $ (0 - \delta, 0 + \delta) \cap S \setminus \{0\} \neq \varnothing, \forall \delta > 0$.
    \item $S = (0,1)$ then $[0,1]$ is the set of cluster points of $S$.
    \item $S = \Q$ then $\R$ is the set of cluster points.
\end{enumerate}

\begin{theorem}
    Let $S \subseteq \R$, then $x$ is a cluster point of $S$ if, and only if, there exists a sequence $\{x_n\}$ of elements in $S \setminus \{x\}$ such that $x_n \to x$.
\end{theorem}

\begin{definition}[Function convergence]
    Consider $S \subseteq \R$, $c$ a cluster point of $S$, and $f: S \to \R$. Then, $f(x)$ converges to $L \in \R$ at $c$ if $\forall \varepsilon > 0, \exists \delta > 0$ such that if $x \in S$ and $0 < |x-c| < \delta$, then $|f(x) - L| < \varepsilon$. \\
    We can write $f(x) \to L$ as $x \to c$ or $\lim_{x \to c} f(x) = L$.
\end{definition}

\begin{theorem}[Uniqueness of the limit of a function]
    Let $c$ be a cluster point of $S \subseteq \R$ and $f: S \to \R$. If $f(x) \to L_1$ and $f(x) \to L_2$ as $x \to c$ then $L_1 = L_2$.
\end{theorem}

\begin{proof}
    Take $\varepsilon > 0$, since $f(x) \to L_1$ and $f(x) \to L_2, \exists \delta_1, \delta_2 \in \R$ such that if $x \in S$ with $0 < |x-c| < \delta_1$ and $0 < |x-c| < \delta_2$ we have $|f(x) - L_1| < \varepsilon/2$ and $|f(x) - L_2| < \varepsilon/2$. \\
    Choose $\delta = \min\{\delta_1, \delta_2\}$. Since $c$ is a cluster point of $S, \exists x_0 \in S$ such that $0 < |x_0-c| < \delta \longrightarrow |L_1-L_2| = |L_1 - f(x_0) + f(x_0) - L_2| \leq |L_1 - f(x_0)| + |f(x_0) - L_2| < \varepsilon$.
\end{proof}

\begin{theorem}
    Consider $c$ a cluster point in $S \subseteq \R$, and $f: S \to \R$. Then, the following statements are equivalent:
    \begin{itemize}
        \item $\lim_{x \to c} f(x) = L$ and,
        \item for every sequence $\{x_n\}$ in $S \setminus \{x\}$ such that $x_n \to c$, then $f(x_n) \to L$.
    \end{itemize}
\end{theorem}

\begin{proof}
    Proving each direction of the theorem individually:
    \begin{enumerate}
        \item Suppose $f(x) \to L$ as $x \to c$, then consider $\{x_n\}$ in $S \setminus \{x\}$ such that $x_n \to c$. Let $\varepsilon > 0, \exists \delta > 0$ such that $|f(x) - L| < \varepsilon$ if $x \in S$ and $0 < |x-c| < \delta$. Since $x_n \to c, \exists N \in \N$ such that $0 < |x_n-c| < \delta, \forall n \geq N$, since $|f(x) - L| \varepsilon, \forall 0 < |x-c| < \delta$ then $f(x_n) \to L$.
        \item Assuming the second part is false, for contradiction, $\exists \varepsilon_0 > 0$ such that $\forall \delta > 0, \eixists x \in S$ such that $0 < |x-c| < \delta$ and $|f(x) - L| \geq \varepsilon_0$. Then, $\forall n \in \n, \exists x_n \in S$ such that $0 < |x_n - c| < 1/n$ and $|f(x_n) - L| \geq \varepsilon_0$. By the squeeze theorem we conclude $x_n \to c$ and $0 = \lim_{n \to \infty} |f(x_n) - L| \geq \varepsilon_0 > 0$ which is a contradiction.
    \end{enumerate}
\end{proof}

\begin{theorem}
    \begin{equation*}
        \lim \limits_{x \to 0} \sin (1/x) \textnormal { does not exist}
    \end{equation*}
\end{theorem}

\begin{proof}
    Let $x_n = \frac{2}{(2n-1)\pi}$. Then, $x_n \neq 0$ and $x_n \to 0$. Now,
    \begin{equation*}
        \sin (1/x_n) = \sin \frac{(2n-1) \pi}{2} = (-1)^{n+1}
    \end{equation*}
\end{proof}

\begin{theorem}
    \begin{equation*}
        \lim \limits_{x \to 0} x \sin(1/x) = 0
    \end{equation*}
\end{theorem}

\begin{proof}
    Suppose $x_n \neq 0$ and $x_n \to 0$. Then,
    \begin{equation*}
        0 \leq |x_n \sin(1/x_n)| = |x_n||\sin(1/x_n)| \leq |x_n|
    \end{equation*}
    By the squeeze theorem, $\lim _{n \to \infty} |x_n \sin(1/x_n)| = 0$
\end{proof}

\begin{theorem}
    Consider $c$ a cluster point in $S \subseteq \R$ and $f,g: S \to \R$, with $f(x) \leq g(x), \forall x \in S$. Suppose $\lim_{x \to c} f(x)$ and $\lim_{x \to c} g(x)$ both exist. Then,
    \begin{equation*}
        \lim \limits_{x \to c} f(x) \leq \lim \limits_{x \to c} g(x)
    \end{equation*}
\end{theorem}

\begin{proof}
    Define $L_1 = \lim_{x \to c} f(x)$ and $L_2 = \lim_{x \to c} g(x)$, and $\{x_n\}$ to be a sequence in $S \setminux \{c\}$ with $x_n \to c$. Then, $f(x_n) \leq g(x_n), \forall n \in \N$. So,
    \begin{equation*}
        L_1 = \lim \limits_{n \to \infty} f(x_n) \leq \lim \limits_{n \to \infy} g(x_n) = L_2
    \end{equation*}
\end{proof}

\begin{definition}[Convergence from the left]
    Consider $c$ to be a cluster point of $S \cap (-\infty, c)$ with $S \subseteq \R$. Then, we say $f(x)$ converges to $L$ from the left (or as $x \to c^-$) if $\forall \varepsilon > 0, \exists \delta >0$ such that if $x \in S$ and $c - \delta < x < c$ we obtain $|f(x) - L| < \varepsilon$. \\
    We denote it by $L = \lim_{x \to c^-} f(x)$.
\end{definition}

\begin{definition}[Convergence from the right]
    Consider $c$ to be a cluster point of $S \cap (c, \infty)$ with $S \subseteq \R$. Then, we say $f(x)$ converges to $L$ from the right (or as $x \to c^+$) if $\forall \varepsilon > 0, \exists \delta >0$ such that if $x \in S$ and $c < x < c + \delta$ we obtain $|f(x) - L| < \varepsilon$. \\
    We denote it by $L = \lim_{x \to c^+} f(x)$.
\end{definition}

\paragraph{Example} Consider
\begin{equation*}
    f(x) = \begin{cases}
    1 \textnormal{ if } x > 0 \\
    0 \textnormal{ if } x < 0
    \end{cases}
\end{equation*}
Then, $\lim_{x \to 0^-} f(x) = 0$ and $\lim_{x \to 0^+} f(x) = 0$, despite $f(0)$ being undefined.

\begin{theorem}
    Consider $c$ a cluster point of $S\cap(-\infty, c)$ and $S\cap(c, \infty)$, with $S \subseteq \R$. Then, $c$ is a cluster point of $S$. Or equivalently:
    \begin{equation*}
        \lim \limits_{x \to c} f(x) = L \longleftrightarrow \lim \limits_{x \to c^-} f(x)= \lim \limits_{x \to c^+} f(x) = L
    \end{equation*}
\end{theorem}

\subsection{Continuity of a function}

As shown in a past example, it is possible that $\lim_{x \to c} f(x) \neq f(c)$. In other words, it is possible that a limit of a function as $x \to c$ (or $x \to c^-$, $x \to c^+$) differs from $f(c)$. Continuity links the two concepts.

\begin{definition}[Continuous function]
    Consider $c \in S \subseteq \R$ a cluster point. We say $f$ is continuous at $c$ if $\forall \varepsilon > 0, \exists \delta > 0$ such that if $x \in S$ with $|x-c| < \delta$ then $|f(x)-f(c)| < \varepsilon$. If $f$ is continuous at all points of $U \subseteq S$ then $f$ is continuous on $U$.
\end{definition}

\begin{theorem}
    Consider $c \in S \subseteq \R$, and $f: S \to \R$, then:
    \begin{enumerate}
        \item if $c$ is not a cluster point of $f$, then $f$ is continuous at $c$,
        \item if $c$ is a cluster point of $f$, then $f$ is continuous at $c$ if, and only if, $\lim_{x \to c} f(x) = f(c)$,
        \item $f$ is continuous at $c$, and only if, for all sequence $\{x_n\}$ in $S$ with $x_n \to c$ then $f(x_n) \to f(c)$.
    \end{enumerate}
\end{theorem}

\begin{proof}
    Proving each statement:
    \begin{enumerate}
        \item Consider $\varepsilon > 0$, since $c$ is not a cluster point of $S$ then $\exists \delta >0$ such that $(c-\delta, c+ \delta) \cap S = \{c\}$, so if $x \in S$ and $|x-c| < \delta$ then $x = c$ and $|f(x) - f(c)| < \varepsilon$.
        \item Proving each direction of the statement:
            \begin{itemize}
                \item $(\Longleftarrow)$ If $\lim_{x \to c} f(x) = f(c)$ then $\forall \varepsilon > 0, \exista \delta > 0$ such that if $x \in S$ and $|x-c| < \delta$ then $|f(x) - f(c)| < \varepsilon$.
                \item $(\Longrightarrow)$ If $f$ is continuous at $c$ then $\forall \varepsilon > 0, \exists \delta > 0$ such that if $x \in S$ with $|x-c| < \delta$ then $|f(x) - f(c)| < \varepsilon$.
            \end{itemize}
        \item Proving each direction of the statement:
            \begin{itemize}
                \item $(\Longrightarrow)$, let $\{x_n\}$ in $S$ with $x_n \to c$. Take $\varepsilon > 0$, since $f$ is continuous at $c$, $\exists \delta > 0$ such that if $|x-c| < \delta$ with $x \in S$ then $|f(x_n) - f(c)| < \varepsilon$. Since $x_n \to c, \exists N \in \N$ such that $|x_n-c| <  \delta, \forall n \geq N$. So, $|x_n-c| < \delta \longrightarrow |f(x_n) - f(c)| < \varepsilon$.
                \item $(\Longleftarrow)$ For contradiction, assume $f$ is not continuous at $c$, then $\exists \varepsilon_0 > 0$ such that $\forall \delta > 0, \exists x \in S$ such $|x-c| < \delta$ and $|f(x) - f(c)| \geq \varepsilon_0$. Thus, $\forall n \in N, \exists x_n \in S$ such that $|x_n-c| < 1/n$ and $|f(x_n) -  f(c)| \geq \varepsilon_0$. Thus, by the squeeze theorem $|x_n - c| \to 0$ and $x_n \to c$ which implies $f(x_n) \to x$ which is a contradiction.
            \end{itemize}
    \end{enumerate}
\end{proof}

\begin{definition}[Bounded function]
    A function $f: S \to \R$ is bounded if $\exists M \geq 0$ such that $|f(x)| \leq M, \forall x \in S$.
\end{definition}

\begin{theorem}
    If $f: [a,b] \to \R$ is continuous at $[a,b]$ then it is bounded.
\end{theorem}

\begin{proof}
    For contradiction, assume $f$ is continuous but not bounded. Then, $\forall n \in \N, \exists x_n \in [a,b]$ such that $|f(x)| \geq n$. By the Bolzano-Weierstrass theorem, there exists a subsequence $\{x_{n_k}\}$ of $\{x_n\}$ with $x \in \R$ and $x_{n_k} \to x$. Since, $x_{n_k} \in [a,b], \forall k \in \N$ then $x \in [a,b]$. Given $f$ is continuous
    \begin{equation*}
        f(x) = \lim \limits_{k \to \infty} f(x_{n_k}) \Longrightarrow |f(x)| = \lim \limits_{k \to \infty} |f(x_{n_k})|
    \end{equation*}
    Hence, $\{|f(x_{n_k})\}$ is bounded and so is $\{n_k\}$ since $n_k \leq |f(x_{n_k})|$, from the definition of a subsequence $k \leq x_k, \forall k \in N$, contradicting the initial claim.
\end{proof}

\begin{theorem}[Min-max theorem or Extreme value theorem]
    Consider $f: [a,b] \to \R$. If $f$ is continuous on $[a,b]$ then it achieves an absolute maximum and absolute minimum on $[a,b]$.
\end{theorem}

\begin{proof}
    For the absolute maximum, if $f$ is continuous then $f$ is bounded. Thus, $E = \{ f(x): x \in [a,b]\}$ is bounded. Let $L = \sup E$ then,
    \begin{itemize}
        \item $L$ is an upper bound for $E$
        \item There exists a sequence $\{f(x_n)\}$ with $x_n \in [a,b]$ such that $f(x_n) \to L$
    \end{itemize}
    By the Bolzano-Weierstrass theorem, there exists a subsequence $\{x_{n_k}\}$ of $\{x_n\}$ and $d \in [a,b]$ such that $x_n \to d$ as $k \to \infty$. Hence,
    \begin{equation*}
        f(d) = \lim \limits_{k \to \infty} f(x_{n_k}) = \lim \limits_{n \to \infty} f(x_n) = L
    \end{equation*}
    by the continuity of $f$ on $[a,b]$. So, $f$ achieves an absolute maximum at $d$.
    The proof for the absolute minimum follows similarly.
\end{proof}

\begin{theorem}
    Consider $f: [a,b] \to \R$. If $f(a) < 0$ and $f(b) > 0$, then $\exists c \in (a,b)$ such that $f(c) = 0$
\end{theorem}

\begin{proof}
    Let $a_1 = a$ and $b_1 = b$. Define $a_n, b_n$ as follows:
    \begin{itemize}
        \item If $f((a_{n-1} + b_{n-1})/2) \geq 0$ then $a_n = a_{n-1}$ and $b_n = (a_{n-1} + b_{n-1})/2$,
        \item If $f((a_{n-1} + b_{n-1})/2 < 0$ then $a_n = (a_{n-1}+b_{n-1})/2$ and $b_n = b_{n-1}$.
    \end{itemize}
    In this way, we obtain:
    \begin{enumerate}
        \item $a \leq a_n \leq a_{n+1} \leq b_{n+1} \leq b_n \leq b, \forall n \in N$,
        \item $b_{n+1} - a_{n+1} = (b_n - a_n)/2, \forall n \in \N$,
        \item $f(a_n) \leq 0$ and $f(b_n) \geq 0, \forall n \in N$.
    \end{enumerate}
    From 1., $\{a_n\}$ and $\{b_n\}$ are bounded and monotone increasing and decreasing respectfully. Thus, $\exists c, d \in [a,b]$ such that $a_n \to c$ and $b_n \to d$. By 2.,
    \begin{equation*}
        b_n - a_n = \frac{b_{n-1} - a_{n-1}}{2} = \frac{1}{4}(b_{n-2} - a_{n-2}) = ... = \frac{1}{2^{n-1}}(b-a)
    \end{equation*}
    And,
    \begin{equation*}
        d - c = \lim \limits_{n \to \infty} (b_n - a_n) = \lim \limits_{n \to \infty} \frac{1}{2^{n-1}} (b-a) = 0 \Longrightarrow d = c
    \end{equation*}
    So, $a_n \to c$ and $b_n \to c$. By 3., $f(c) = \lim_{n \to \infty} f(a_n) \leq 0$ and $f(c) = \lim_{n \to  \infty} f(b_n) \geq 0$. Therefore, $f(c) = 0$.
\end{proof}