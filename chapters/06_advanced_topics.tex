\chapter{Advanced topics}

\section{Sequences of functions}

\begin{definition}[Power series]
    A power series about $x_0$ is series of the form:
    \begin{equation*}
        \sum \limits_{n = 0}^\infty a_n(x - x_0)^n
    \end{equation*}
\end{definition}

\begin{theorem}
    Suppose $\{|a_n|^{1/n}\}$ converges, \emph{i.e.}:
    \begin{equation*}
        R = \lim \limits_{n \to \infty} |a_n|^{n/n}
    \end{equation*}
    and define $p$, the \emph{radius of convergence}, as:
    \begin{equation*}
        p = \begin{cases}
            \frac{1}{R} \text{ if } R > 0 \\
            \infty \text{ if } R = 0
        \end{cases}
    \end{equation*}
    Then:
    \begin{enumerate}
        \item If $|x-x_0| < p$, the series $\sum a_n(x - x_0)^n$ converges;
        \item If $|x-x_0| > p$, the series diverges.
    \end{enumerate}
\end{theorem}

\begin{proof}
    First, notice that:
    \begin{equation*}
        \lim \limits_{n \to \infty} |a_n(x-x_0)^n|^{1/n} = R |x-x_0|
    \end{equation*}
    So, the theorem is valid from the Root test.
\end{proof}

Suppose $\sum a_n(x-x_0)^n$ is a power series with radius of convergence $p$. Then, define $f: (x_0 - p, x_0 + p) \to \R$ such that:
\begin{equation*}
    f(x) := \sum \limits_{n=0}^\infty a_n (x - x_0)^n
\end{equation*}

So, $f(x)$ is a limit of a sequence of functions $f_n(x)$, \emph{i.e.}:

\begin{equation*}
    f(x) = \lim \limits_{m \to \infty} f_m(x)
\end{equation*}

Where,

\begin{equation*}
    f_m(x) = \sum \limits_{n=0}^m a_n(x-x_0)^n
\end{equation*}

The definition of a function in such form leads to a number of questions:
\begin{enumerate}
    \item Is $f(x)$ continuous?
    \item Is $f(x)$ differentiable, and does $f'(x) = \lim_{n \to \infty} f'_m(x)$?
    \item If $f(x)$ is continuous, does it mean:
        \begin{equation*}
            \int_a^b f(x) \dint x = \lim \limits_{n \to \infty} \int_a^b f_m(x) \dint x \text{ ?}
        \end{equation*}
\end{enumerate}

In order to answer these questions, some tools must be build first.

\subsection{Pointwise and uniform convergence}

We start from framework more general than power series.

\begin{definition}[Pointwise convergence]
    Define $f:S \to \R$ and $f_n: S \to \R$ with $n \in \N$. The sequence of functions $\{f_n\}$ is said to convergence pointwise to $f$ if:
    \begin{equation*}
        \lim \limits_{n \to \infty} f_n(x) = f(x), \forall x \in S
    \end{equation*}
\end{definition}

\paragraph{Examples}

\begin{enumerate}
    \item Consider $f_n(x) = x^n$ on $[0,1]$. Then,
        \begin{equation*}
            \lim \limits_{n \to \infty} f_n(x) = \begin{cases}
                0 \text{ if } x \in [0,1) \\
                1 \text{ if } x=1
            \end{cases}
        \end{equation*}
        Thus $\{f_n(x)\}$ converges pointwise to the function above, hence a sequence of continuous function may not converge to a continuous function.
    \item Consider $f_n(x) = \sum_{m=0}^n x^m$ for $x \in (-1,1)$. Then,
        \begin{equation*}
            \lim \limits_{n \to \infty} f_n(x) = \lim \limits_{n \to \infty} \sum \limits_{m=0}^n x_m = \frac{1}{1-x}
        \end{equation*}
\end{enumerate}

\begin{definition}[Uniform convergence]
    For $n \in \N$, define $f_n: S \to \R$ and $f: S \to \R$. Then, $f_n$ \emph{converges uniformly} to $f$ if $\forall \varepsilon > 0, \exists N \in \N$ such that $|f_n(x) - f(x)| < \varepsilon, \forall n \geq N, \forall x \in S$.
\end{definition}

\begin{theorem}
    Consider $f_n: S \to \R$ and $f: S \to \R$. Then, if $f_n$ converges to $f$ uniformly it also converges to $f$ pointwise.
\end{theorem}

\begin{proof}
    Consider $c \in S$ and let $\varepsilon > 0$. Then, if $f_n \to f$ uniformly, $\exists N \in \N$ such that $|f_n(x) - f(x)| < \varepsilon, \forall n \geq N, \forall x \in S$. So, $\lim_{n \to \infty} f_n(c) = f(c), \forall c \in S$. Therefore, $f_n$ converges to $f$ pointwise.
\end{proof}

\begin{theorem}[Weierstrass M-test]
    Define $f_i: S \to \R$ and suppose $\exists M_i > 0$ such that:
    \begin{itemize}
        \item $|f_i(x)| < M_i, \forall x \in S$,
        \item $\sum_{i=1}^\infty M_i$
    \end{itemize}
    Then,
    \begin{enumerate}
        \item $\sum_{i=1}^\infty f_i(x)$ converge absolutely for all $x \in S$,
        \item Define $f(x) = \sum_{i = 1}^\infty f_i(x), \forall x \in S$, then $\sum_{i=1}^n f_i$ converges uniformly to $f$ on $S$.
    \end{enumerate}
\end{theorem}

\begin{proof}
    Proving each statement:
    \begin{enumerate}
        \item The first item implies the sequence $\{|f_i(x)|\}_i$ is bounded for all $x \in S$ by $M_i$. So,
            \begin{equation*}
                \sum \limits_{i=1}^n |f_i(x)| \leq \sum \limits_{i=1}^n M_i
            \end{equation*}
        By the comparison test, the sequence $\{\sum_{i=1}^n |f_i(x)|\}_n$ converges absolutely, since $\sum_{i=1}^\infty M_i$ converges. Hence, $\sum_{i=1}^\infty f_i(x)$ converges absolutely for all $ x \in S$.
        \item Let $\varepsilon > 0$. Since $\sum M_i$ converges, then $\exists N_0 \in \N$ such that:
            \begin{equation*}
                \sum \limits_{i = n+1}^\infty M_i = \left |
                    \sum \limits_{i = 1}^\infty M_i
                    - \sum \limits_{i = 1}^n M_i
                \right | < \varepsilon, \forall n \geq N_0
            \end{equation*}
            Take $N = N_0$. Then,
            \begin{align*}
                \left |
                    f(x) - \sum \limits_{i = 1}n f_i(x)
                \right |
                &= \left |
                    \sum \limits_{i = n+1}^\infty f_i(x)
                \right | \\
                &\leq \sum \limits_{i = n+1}^\infty |f_i(x)| \\
                &\leq \sum \limits_{i = n+1}^\infty M_j \\
                &< \varepsilon, \forall n \geq N, \forall x \in S
            \end{align*}
            So, $\sum_{i=1}^n f_i(x) \to f(x), \forall x \in S$.
    \end{enumerate}
\end{proof}