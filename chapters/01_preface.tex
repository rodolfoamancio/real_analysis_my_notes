\chapter{Preface}

First, let me be clear. I am not a mathematician. These notes are not intended as a manual, however I like to teach, explain science, and I am firm believer that the best way to learn something is to teach each. Richard Feynman famously said the best way to learn is to follow these steps:

\begin{enumerate}
    \item Study: arguably the easiest part, whether you like to take notes on paper, tablet or your computer. Whether you like to sit on a table or on the couch. However, as many people who have come this far know, reading, taking notes or making exercises only take you so far.
    \item Teach: this is where the fun begins, as you try to explain something you know (or think you know) to someone, you start being more aware of your limitations, of the gaps in the proofs you cannot explain, the unexpected questions that may appear lead you astray. Even without an audience, this is a nice thing to do as it forces you to be clear and think about how to explain something in a clear yet rigorous way.
    \item Fill the gaps: now it is time to come back to studying, reading more, exploring new books, papers or what else. Once you have discovered your limitations on the previous step, you are once again in the position to learn and study, but now you know where to look.
    \item Simplify: one of the greatest sins we commit is to get stuck with fancy proofs, delude ourselves in the beauty of math. Make it so that people will understand and enjoy what they are reading or listening.
\end{enumerate}

In this document, I have written my learnings from studying Real Analysis. I hope to learn more while writing it.