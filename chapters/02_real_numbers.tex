\chapter{The real numbers}

During high school math we are often given a simplified definition of the real numbers, one it may take a while to fully grasp how awkward it is: "The real numbers is the set which contains the rational numbers and the irrational numbers". Taking alone it may seem a reasonable statement. In fact, it is true. However, if we start with the natural numbers there is a very concise and clear way of writing it:

\begin{equation}
    \N = \{1, 2, 3, ...\}
\end{equation}

Taking one step further, the integers follow quite naturally:

\begin{equation}
    \Z = \{..., -3, -2, -1, 0, 1, 2, 3, ... \}
\end{equation}

And even for the rationals, we can clearly write:

\begin{equation}
    \Q = \left \{
        \frac{p}{q}: p, q \in \Z \text{ and } q \neq 0
    \right \}
\end{equation}

Now, for the real numbers things are not so clear. So, we are stuck with our initial understanding of the big set which includes the rational and irrational numbers. And as the saying goes "The devil in the details". Calculus, with all of its beautiful and powerful tools: limits, derivatives and integrals, relies fundamentally on real numbers. In order to rigorously work with its results, investigate the extremes and challenging cases, and proving its results depends on a formal definition for the real numbers.

So, let's start by looking more carefully at this similarly weird creature.

\section{Irrational numbers}

Before we proceed, let's take a minute to appreciate why we need irrational numbers. The following result will play an important role to distinguish the "holes" of the rational numbers when compared with the reals. We begin with a theorem.

\begin{theorem}
    There is no such number whose square root is $2$
\end{theorem}

\begin{proof}
    As stated before, a rational number is one that can be written in the form $p/q$, with $q \neq 0$. Our approach here is what is called proof by contradiction. We will assume the opposite of what we want to prove, once we arrive at some absurd result we will conclude our initial assumption was wrong. Therefore, assume $\exists p, q \in \Z: (p/q)^2 = 2$, additionally, we take $p$ and $q$ with no common factors, such that the fraction $p/q$ is written in its simplest form. \\
    If this is true, we can rearrange the relation into: $p^2 = 2 q^2$. Which implies $p^2$ is an even number, since the square of any odd number is odd, $p$ must also be even, \emph{i.e.} $p = 2r$. \\
    Now, replacing $p$ on the previous equation yields $4 r^2 = 2 q^2 \Rightarrow q^2 = 2 r^2$, which implies $q^2$ and so is $q$. \\
    This directly contradicts our initial assumption, since $p$ and $q$ are both even from the result above. Hence, our initial assumption must be wrong, and we conclude $\nexists p,q \in \Z: (p/q)^2=2$.
\end{proof}

In order to deal with irrational numbers, the set of real numbers is the natural extension necessary. Before we deal with it in a rigorous way, will start with the necessary tools to help us on this journey.

\section{Preliminaries}

This section aims to define some basic definitions and results that will help us to deal with real numbers, and the other topics of interest.

\subsection{Set theory}

\begin{definition}[Set]
    A set is a collection of objects, called elements or members. An empty set is a set with no elements, denoted $\varnothing$.
\end{definition}

Usually, we write a set $A$ as $A = \{ a_1, a_2, ...\}$ where $a_1, a_2, ...$ are the elements of the set. Some important notations are:

\begin{itemize}
    \item $a \in A$: meaning $a$ is an element of $A$
    \item $a \notin A$: meaning $a$ is not an element of $A$
    \item $\forall$: meaning `for all'. For example, in mathematical notation the expression `for all $a$ which is an element of $A$' would be $\forall a \in B$
    \item $\exists$: meaning `there exists'. The opposite would be $\nexists$
    \item $\Rightarrow$: implies
    \item $\Leftrightarrow$: if and only if
\end{itemize}

On a side note, the terms `implies' and `if and only if' have fundamental differences which will lead to different approaches in demonstrations and results. For instance, let's say proposition $P_1$ implies proposition $P_2$. In mathematical notation $P_1 \Rightarrow P_2$. This means that if $P_1$ is true, so is  $P_2$, but it does not say anything about the opposite direction. That is, if $P_2$ is true, not necessarily $P_1$ is true. On the other hand when the relation is $P_1$ is true if, and only if, $P_2$ is true. Or, $P_1 \Leftrightarrow P_2$ then the relation works both ways: if $P_1$ is true, so is $P_2$ and if $P_2$ is true, so is $P_1$. During proofs, when we have a $\Leftrightarrow$ relation, the result must be proven in both directions.

\begin{definition}[Subset]
    $A$ is a subset of $B$ if, every element of $A$ is also an element of $B$. Notation: $A \subseteq B$. Equivalently, if $B$ is a superset of $A$, it is denoted $B \supseteq A$.
\end{definition}

Informally, we understand that two sets are equal if every element of one set is also an element of the other, and vice-versa. On mathematical notation:

\begin{definition}[Equal sets]
    Two sets, $A$ and $B$, are equal if $A \subseteq B$ and $B \subseteq A$. Hence, $A = B$.
\end{definition}

\begin{definition}[Proper subset]
    A set $A$ is a proper subset of $B$ if $A \subseteq B$ and $A \neq B$. Notation: $A \subsetneq B$.
\end{definition}

Now, tow (or more) sets can be combined by operations. We define:

\begin{itemize}
    \item Union: $A \cup B = \{ x: x \in A \text{ or } x \in B\}$
    \item Intersection: $A \cap B = \{ x: x \in A \text{ and } x \in B\}$
    \item Difference: $A \setminus B = \{ x : x \in A \text { and } x \notin B\}$
    \item Complement: $A^C = \{ x \notin A \}$
\end{itemize}

\begin{definition}[Disjoint sets]
    Two sets, $A$ and $B$, are disjoint if $A \cap B = \varnothing$.
\end{definition}

Some important from set theory are the so-called De Morgan's laws:

\begin{theorem}[De Morgan's Laws]
    If $A$, $B$ and $C$ are sets, then:
    \begin{itemize}
        \item $(B \cup C)^C = B^C \cap C^C$
        \item $(B \cap C)^C = B^C \cup C^C$
        \item $A \setminus (B \cup C) = (A \setminus B) \cap (A \setminus C)$
        \item $A \setminus (B \cap C) = (A \setminus B) \cup (A \setminus C)$
    \end{itemize}
\end{theorem}

Following the proof of the first result is shown, the other results can be derived similarly.

\begin{proof}
    Two sets $X$ and $Y$ are equal if $X \subseteq Y$ and $Y \subseteq X$. Our goal is to show that $(B \cup C)^C \subseteq B^C \cap C^C$ and $(B \cup C)^C \supseteq B^C \cap C^C$. \\
    Let $x \in (B \cup C)^C$. It follows that $x \notin (B \cup C)$ then $x \notin B$ and $x \notin C$. So, $x \in B^C \cap C^C$ and we have $(B \cup C)^C \subseteq B^C \cap C^C$. \\
    From the opposite direction, let  $x \in B^C \cap C^C$. Then $x \in B^C$ and $x \in C^C$ which means $x \notin B$ and $x \notin C$. So $x \notin (B \cup C) \Rightarrow x \in (B \cup C)^C$. So $B^C \cap C^C \subseteq (B \cup C)^C$.
    Since $(B \cup C)^C \subseteq B^C \cap C^C$ and $B^C \cap C^C \subseteq (B \cup C)^C$, we have $(B \cup C)^C \subseteq B^C \cap C^C$.
\end{proof}

\subsubsection{Fields}

\begin{definition}[Field]
    A set $F$ is a field if satisfies the following properties:
    \begin{itemize}
        \item For addition
            \begin{enumerate}
                \item If $x, y \in F \Rightarrow x + y \in F$
                \item Commutativity: $\forall x, y \in F: x + y = y + x$
                \item Associativity: $\forall x, y, z \in F: (x+y) + z = x + (y + z)$
                \item Additive identity: $\exists 0 \in F: 0 + x = x, \forall x \in F$
                \item Additive inverse: $\exists -x \in F: x + (-x) = 0, \forall x \in F$
            \end{enumerate}
        \item For multiplication
            \begin{enumerate}
                \item If $x, y \in F \Rightarrow x \cdot y \in F$
                \item Commutativity: $\forall x, y \in F: x \cdot y = y \cdot x$
                \item Associativity: $\forall x,y, z \in F: (x\cdot y) z = x (y \cdot z)$
                \item Multiplicative identity: $\exists 1 \in F: x \cdot 1 = x, \forall x \in F$
                \item Multiplicative inverse: $\exists x^{-1} \in F: x \cdot x^{-1} = 1, \forall x \in F$
            \end{enumerate}
    \end{itemize}
\end{definition}

\begin{theorem}
    If $F$ is a field, $\forall x \in F: x \cdot 0 = 0$.
\end{theorem}

\begin{proof}
    If $x \in F$ then $0x \in F$ so $0 = 0x + (-0x) = 0x + 0x + (-0x) = 0x$.
\end{proof}

\begin{definition}[Ordered field]
    An ordered field $F$ is a field which satisfies $\forall x, y, z \in F$:
    \begin{enumerate}
        \item If $x < y$ then $x + z < y + z$
        \item If $x > 0$ and $y > 0$ then $xy > 0$ 
    \end{enumerate}
\end{definition}

\subsubsection{Bounds}

\begin{definition}[Bounds]
    Let $A \subseteq B$. Then,
    \begin{enumerate}
        \item If $\exists u \in B: u \geq a, \forall a \in A$ then $A$ is bounded above and $u$ is an upper bound for $A$.
        \item If $\exists l \in B: l \leq a, \forall a \in A$ then $A$ is bounded below and $l$ is a lower bound for $A$.
    \end{enumerate}
\end{definition}

\begin{eg}
    Consider the set $B = \R$ and $A = [0,1]$. Then, $2, 2.5, \pi$ are all upper bounds for $A$. Similarly, $-1, 0, -\pi$ are all lower bounds for $A$.      
\end{eg}

\begin{definition}[Supremum]
    Let $A \subseteq B$ with $A$ bounded above. Then $s$ is the least upper bound (or supremum) if:
    \begin{enumerate}
        \item $s$ is an upper bound for $A$, and
        \item If $u$ is another upper bound for $A$ then $s \leq u$.
    \end{enumerate}
    Mathematically, we write $s = \sup A$.
\end{definition}

\begin{definition}[Infimum]
    Let $A \subseteq B$, with $A$ bounded below. Then, $i$ is the greatest lower bound (or infimum) of $A$ if:
    \begin{enumerate}
        \item $i$ is a lower bound for $A$, and
        \item If $l$ is another lower bound for $A$ then $i \geq l$.
    \end{enumerate}
    Mathematically, we write $i = \inf A$.
\end{definition}

\begin{eg}
    Consider $B = \R$ and $A = (0,1) \subseteq A$. Then $1, \pi, 10$ are all upper bounds for $A$ but $1$ is the least upper bound (or infimum). On the other hand, $-10, -1, 0$ are all lower bounds for $A$, but only $0$ is the greatest lower bound (or infimum) of $A$.
\end{eg}

The previous example shows an important characteristic of the supremum (or infimum). In this case $0 \notin A$ and $1 \notin B$. We can also define:

\begin{definition}[Maximum]
    Let $A$ be a set bounded above, then $M$ is the maximum of $A$ if $M \in A$ and $M \geq a, \forall a \in A$.
\end{definition}

\begin{definition}[Minimum]
    Let $A$ be a set bounded below, then $m$ is the minimum of $A$ if $m \in A$ and $m \leq a, \forall a \in A$.
\end{definition}

\begin{remark}
    Notice that a set may have an infimum and not a minimum, as the previous example, since $0 \notin A$. The same result is valid for the supremum and maximum. On the other hand, if a set $A$ has a maximum, then it necessarily has a supremum. An equivalent result holds for the infimum and minimum.
\end{remark}

\subsection{Function}

The formal definition of function is the following:

\begin{definition}[Function]
    Given a set $A$ and a set $B$, a function is a mapping rule which takes as an argument an element $a \in A$ and associates it with an element of $B$. We write $f: A \to B$. $f(a)$ is used to express the element of $B$, $f(a) \in B$, associated with the element $a \in A$. $A$ is called the domain of the function, while $B$ is its codmoain. The image of $f$ is not necessarily equal to $B$, but refers to $\{b \in B: b = f(a) \text{ for some } a \in A\} \subseteq B$.
\end{definition}

It is worth noting how this definition liberates math from the usual `formula' understanding of a function. In particular, this definition is closer to Dirichlet's definition, and it allows math to deal with more interesting and complex functions, such as:

\begin{eg}[Dirichlet's function]
    \begin{equation*}
        f(x) = \begin{cases}
            1 \text{ if } x \in \Q \\
            0 \text{ if } x \notin \Q
        \end{cases}
    \end{equation*}
\end{eg}

\vspace{1em}
This broader definition of function will lead to interesting results and test some limits in maths. But more on that latter.

\subsubsection{Classification}

\begin{definition}[Injective function]
    The function $f: A \to B$ is called 1-1 or injective if $f(a_1) = f(a_2) \Rightarrow a_1 = a_2$. Equivalently, $a_1 \neq a_2 \Rightarrow f(a_1) \neq f(a_2)$.
\end{definition}

\begin{definition}[Surjective function]
    The function $f: A \to B$ is called onto or surjective if $\forall b \in B, \exists a \in A$ such that $ f(a) = b$.
\end{definition}

\begin{definition}[Bijective function]
    A function that is both injective and surjective is called bijective.
\end{definition}

\subsubsection{Composition and inverse}

\begin{definition}[Composite function]
    If $f: A \to B$ and $g: B \to C$, then $f \circ g: A \to C$ is defined by $(f \circ g)(x) = g(f(x))$.
\end{definition}

\begin{definition}[Inverse function]
    Consider $f: A \to B$ a bijective function. Then the inverse function $f^{-1}: B \to A$ is defined by: if $b \in B$ then $f^{-1}(b) \in A$ is the unique element $f^{-1}(b)$ such that $f(f^{-1}(b)) = b$.
\end{definition}

\subsection{The absolute function}

The absolute function plays an important role in the proofs and arguments that are to come. First, it is defined as:

\begin{definition}[Absolute function]
    The absolute function $f(x): \R \to \R_+$ is defined as:
    \begin{equation}
        |x| = \begin{cases}
            x \: \: \:\textnormal{ if } x \geq 0, \\
            -x \textnormal{ if } x < 0
        \end{cases}
    \end{equation}
\end{definition}

\vspace{1em}
It leads to a very important result, called the triangle inequality:

\begin{theorem}[Triangle inequality]
    $\forall a, b \in \R$, $|a+b| \leq |a| + |b|$
\end{theorem}

\begin{proof}
    Let $x, y \in \R$. Then, $x + y \leq |x| + |y|$ and
    \begin{equation*}
        (-x) + (-y) \leq |-x|+|-y| = |x| + |y|
    \end{equation*}
    Hence, $-(|x|+|y|) \leq x+y \leq |x| + |y|$ and we obtain
    \begin{equation*}
        |x+y| \leq |x| + |y|
    \end{equation*}
\end{proof}

\subsection{Induction}

The natural numbers have a property which leads to very important applications. This can be enunciated as:

\subsubsection{Well ordering property of $\N$}

If $S \subseteq \N$ and $S \neq \varnothing$. Then, $\exists x \in S$ such that $x \leq y, \forall y \in S$. \\

An important tool that arises from it is called `Induction'. We can state it as:

\subsubsection{Proof by induction}

Let $P(n)$ be a statement depending on $n \in \N$. Assume:

\begin{enumerate}
    \item Base case: $P(1)$ is true
    \item Inductive case: If $P(m)$ is true, so is $P(m+1)$.
\end{enumerate}

From it, we conclude $P(n)$ is true for all $n \in \N$.

\begin{eg}
    Prove that 
    \begin{equation*}
        1 + c + c^2 + ... + c^n = \frac{1-c^{n+1}}{1-c}, \forall c \neq 1, \forall n \in \N
    \end{equation*}
    
    using induction. \\
    
    Following the algorithm presented before:
    \begin{enumerate}
        \item Base case.
            \begin{equation*}
                1 + c = \frac{1-c^2}{1 - c} = \frac{(1-c)(1+c)}{1-c} = 1 + c
            \end{equation*}
            As expected.
        \item Inductive case.
            Assume
            \begin{equation*}
                1 + c + c^2 + ... + c^m = \frac{1-c^{m+1}}{1-c}
            \end{equation*}
            is true. Now, for $m+1$:
            \begin{equation*}
                \begin{split}
                    1 + c + c^2 + ... + c^{m+1} & = (1 + c + c^2 + ... +c^{m}) + c^{m+1} \\
                    & = \frac{1-c^{m+1}}{1-c} + c^{m+1} \\
                    & = \frac{1-c^{m+1} + c^{m+1} + c^{m+2}}{1-c} \\
                    & = \frac{1-c^{m+2}}{1-c}
                \end{split}
            \end{equation*}
            Hence, the relation still holds.
    \end{enumerate}
\end{eg}

\section{Defining $\R$}

\subsection{The incompleteness of $\Q$}

Now, let's revisit our initial problem, namely $\sqrt{2} \notin \Q$. First, we start with a theorem:

\begin{theorem}
    The set $E = \{ x \in \Q: 0 < x < \sqrt{2}\}$ is bounded above and does not have a supremum in $\Q$.
\end{theorem}

\begin{proof}
    First, consider $q \in \Q$ then $q^2 <  2 < 4 \Rightarrow q^2 - 4 < 0 \Rightarrow (q-2)(q+2) < 0$. Since $q > 0$ we have $q - 2 < 0 \Rightarrow q <  2$. Hence, $2$ is an upper bound for $E$.\\
    Next, to show that $\nexists \sup E \in \Q$ we begin by assuming $x = \sup E \in Q$. \\
    Assume, for contradiction, $x^2 < 2$. Define
    \begin{equation*}
        h = \min \left\{ 
            \frac{1}{2}, \frac{2 - x^2}{2(2x+1)}
        \right\} < 1
    \end{equation*}
    Then, $h > 0$. Now we prove $h + x \in E$. Computing $(x+h)^2 = x^2 +2xh + h^2 < x^2 + 2xh + h$ since $h < 1$. So
    \begin{equation*}
        \begin{split}
            (x+ h)^2 < x^2 + (2x + 1)h &= x^2 + (2x+1)\frac{2-x^2}{2(2x+1)} \\
            &= x^2 + 2 - x^2 \\
            &= 2
        \end{split}
    \end{equation*}
    Therefore $(x+h)^2 < 2$ which implies $x + h \in E$ and $x + h > x$ so $x \neq \sup E$ which is a contradiction. Therefore, $x^2 \geq 2$. \\
    Now, assume for contradiction $x > 2$. Then, define
    \begin{equation*}
        h = \frac{x^2 - 2}{2x}
    \end{equation*}
    Note that $x^2 > 2 \Rightarrow h > 0 \Rightarrow x - h < x$. Now we prove $x-h$ is an upper bound for $E$. Compute $(x-h)^2 = x^2 -2xh + h^2 = x^2 - (x^2 - 2) + h^2 = 2 + h^2 > 2$. Let $q \in E$, \emph{i.e.} $0 < q < \sqrt{2}$. Then, $q^2 < 2 < (x - h)^2 \Rightarrow 0 <  (x - h)^2 - q^2 \Rightarrow 0 < (x-h+q)(x-h-q)$ and
    \begin{equation*}
        0 < \left(
            \frac{x^2+2}{2x} + q
        \right)(x-h-q).
    \end{equation*}
    Since $q > 0$ and $(x^2+2)/(2x) > 0$ then $0 < x-h-q \Rightarrow q < x-h$. Thus, $\forall q \in E, q < x - h \Rightarrow x-h$ is an upper bound for $E$. Since $x = \sup E \Rightarrow x \leq x + h \Rightarrow h \leq 0$, which is a contradiction. \\
    Thus, $x^2 = 2$ and $x > 1$. \\
    For contradiction, assume $\exists m,n \in \N$ such that $m > n, x = m/n$. Then, $\exists n \in \N$ such that $nx \in \N$. Let $S = \{ k \in \N: kx \in \N\}$, note that $n \in S \Rightarrow S \neq \varnothing$. By the well-ordering of $\N$, $S$ has the least element $k_0 \in S$. Define $k_1 = k_0x-k_0 \in \Z$. Since $x > 1, k_1 = k_0(x-1) > 0 \Rightarrow k_1 \in \N$. Since $x^2 = 2 \Rightarrow 4 - x^2 > 0 \Rightarrow (2-x)(2+x) > 0 \Rightarrow 2-x > 0 \Rightarrow x < 2$. Then $k_1 = k_0(x-1) < k_0(2-x) = k_0$. Thus, $k_1 \in \N$ and $k_1 < k_0$. Computing $xk_1 = x(xk_0 -k_0) = x^2k_0 - xk_0 = 2k_0 - xk_0 = k_0 + (k_0 - xk_0) = k_0 - k_1 \in \N$. Thus, $k_1 \in S$ and $k_1 < k_0$ which means $k_0$ is not the least element in $S$ and $\sup E$ does not exist in $\Q$.
\end{proof}

\subsection{The definition of $\R$}

First, in order to define $\R$ the previous result about the lack of an upper bound for $E = \{ q \in \Q: 0 < q < \sqrt{2}\}$ allows us to introduce a definition.

\begin{definition}[Least upper bound property]
    An ordered set $S$ has the least upper bound property if every nonempty and bounded above subset $E \subseteq S$ has a supremum in $S$.
\end{definition}

The previous definition could be stated about a `Greatest upper bound property'. Clearly $\Q$ does not have the Least upper bound property as the previous subsection has shown.

Now, for the real numbers,

\begin{theorem}[Existence of $\R$]
    There exists a unique ordered field which contains $\Q$ and has the least upper bound property. This field is denoted by $\R$.
\end{theorem}

\begin{theorem}
    There exists a unique $r \in \R$ such that $r>0$ and $r^2 = 2$.
\end{theorem}

\begin{proof}
    First, let $\tilde{E} = \{ x \in \R: 0 < x < \sqrt{2} \}$. Then, $\tilde{E}$ is bounded above. Take $r = \sup \tilde{E}$. The same proof as before show $r > 1$ and $r^2 = 2$. We now prove $r$ is unique. Suppose $\tilde{r} \in \R, \tilde{r}>0$ and $\tilde{r}^2 = 2$. Then, $0 = \tilde{r}^2 - r^2 = (\tilde{r} - r)(\tilde{r} + r) \Rightarrow 0 = \tilde{r} - r \Rightarrow r = \tilde{r}$.
\end{proof}

\subsection{The density of $\Q$ in $\R$}

The set $\Q$ contains $\N$, and $\R$ contains $\Q$. The following theorems shows how $\N$ and $\Q$ sit inside $\R$:

\begin{theorem}[Archimedean property]
    If $x, y \in \R$ and $x > 0$ then $\exists n \in \N$ such that $nx < y$.
\end{theorem}

\begin{proof}
    Suppose $x, y \in \R$ and $x > 0$. We need to show $\exists n \in \N$ such that $nx < y$, \emph{i.e.} $x < y/n$. \\
    Assume for contradiction $\forall n \in \N: n \leq y/x$. Then $\N \subseteq \R$ is bounded above, hence it has a supremum, by the least upper bound property of $\R$ with value $a \in \R$. Since $a$ is the supremum of $\N$ then $a-1$ is not an upper bound for $\N$. Therefore, $\exists m \in \N$ such that $a-1 < m \Rightarrow a < m + 1$ which implies $a$ is not an upper bound for $\N$ contradiction our initial claim.
\end{proof}

\begin{theorem}[Densit of $\Q$ in $\R$]
    If $x, y \in \R$ and $x < y$ then $\exists q \in \Q$ such that $x < r < y$.
\end{theorem}

\begin{proof}
    Let $x, y \in \R$ and $x < y$, then:
    \begin{enumerate}
        \item If $x < 0 < y$ we have $r = 0 \in \Q$.
        \item If $0 \leq x < y$ then by the Archimedean property, $\exists n \in \N$ such that $n(y-x) > 1$ and $\exists l \in \N$ such that $l > nx$. Thus, $S = \{ k \in \N: k > nx\} \neq \varnothing$. By the well ordering property of $\N$, $S$ has the least element $m$.\\
        Since $m \in S \Rightarrow nx < m$. Since $m$ is the least element of $S$, $m-1 \notin S \Rightarrow m-1 \leq nx \Rightarrow m \leq nx + 1$. Thus, $nx < m < nx + 1 \Rightarrow x < m/n < y$. So, $r = m/n \in \Q$ is the solution.
        \item If $x < y \leq 0$, then $0 \leq -y < x$. Then by the previous result $\exists \tilde{r} \in \Q$ such that $-y < \tilde{r} < -x$, or equivalently $x < -\tilde{r} < y$ and $r = -\tilde{r}$ is the solution.
    \end{enumerate}
\end{proof}

\section{Cardinality}

Now, we turn our attention to cardinality, which is an approach to compare the size of sets.

\begin{definition}[Cardinality]
    Two sets, $A$ and $B$, have the same cardinality if there exists a bijective function $f: A \to B$.
\end{definition}

\paragraph{Notation}

\begin{itemize}
    \item If $A$ and $B$ have the same cardinality we write $|A| = |B|$
    \item If $|A| = |\{ 1, 2, 3, ..., n\}|$ we write $|A| = n$
    \item If there exists an injective function $f: A \to B$ we write $|A| \leq |B|$
    \item If $|A| \leq |B|$ and $|A| \neq |B|$ then $|A| < |B|$
\end{itemize}

\begin{theorem}[Cantor-Schorer-Bernstein]
    If $|A| \leq |B|$ and $|B| \leq |A|$ then $|A| = |B|$
\end{theorem}

\subsection{Countable, uncountable and countably infinite sets}

\begin{definition}[Countably infinite]
    If $|A| = |\N|$ then $A$ is countably infinite.
\end{definition}

\begin{definition}[Countable sets]
    If $A$ is countably infinite or finite, then $A$ is countable.
\end{definition}

\begin{definition}[Uncountable set]
    If $A$ is neither countably infinite nor finite, then $A$ is uncountable.
\end{definition}

Since both cardinality and countability have been introduced it is time to appreciate some very interesting results.

\epigraph{There are twice as many numbers as numbers}{\textit{Richard Feynman}}

\begin{theorem}
    The set of positive even numbers is countable, \emph{i.e.} $|\{ 2\times n: n \in \N\}| = |\N|$. And so are the odd numbers, $|\{ 2\times n - 1: n \in \N\}|$.
\end{theorem}

\begin{proof}
    Let $f: \N \to \{2 \times n: n \in \N \}$. So, $f(n) = 2n, \forall n \in \N$. \\
    First, if $f(n_1) = f(n_2)$ then $2n_1 = 2n_2$, hence $n_1 = n_1$ and $f$ is injective. \\
    Second, let $m \in \{ 2\times k: k \in \N\}$. Then, $\exists n \in \N$ such that $m = 2n$, the function $f(n) = 2n = m \Rightarrow n = m/2$. Therefore, $f$ is also surjective, and by result there exists a bijective function $f: \N \to \{2 \times n: n \in \N \}$, and we conclude $|N| = |\{ 2\times n: n \in \N\}|$. \\
    The proof for the odd numbers is similar.
\end{proof}

\begin{theorem}[Countability of $\Z$]
    The set of integers is countable, \emph{i.e.} $|\N| = |\Z|$.
\end{theorem}

\begin{proof}
    Define $f: \N \to \Z$ as
    \begin{equation*}
        f(n) = \begin{cases}
            \frac{n-1}{2} \text{ if } n \text{ is odd} \\
            \frac{-n}{2} \text{ if } n \text{ is even}
        \end{cases}
    \end{equation*}
    $f(n)$ is both injective and surjective. Hence, $|\Z| = |\N|$.
\end{proof}

\begin{theorem}[Countability of $\Q$]
    $|\Q| = |\N|$
\end{theorem}

\begin{proof}
    Set $A_1 = \{0 \}$, for $n \geq 2$ define $A_n = \{ \pm p/q: p,q \in \N \text{ and are in the lowest terms with } p + q = n\}$. For example, $A_1 = \{0\}$, $A_2 =\{ 1/1, -1/1\}$, $A_3 = \{ 1/2, -1/2, 2/1, -2/1\}$, and so on.\\
    Since each rational number appear in only one $A_n$ and every rational number can be represented by the relation above, $|\Q| = |\N|$.
\end{proof}

\begin{theorem}
    If $A \subseteq B$ and $B$ is countable, then $A$ is either countable or finite.
\end{theorem}

\begin{theorem}
    $ $ \newline
    \begin{enumerate}
        \item If $A_1, A_2, ..., A_m$ are each countable sets, then $\bigcup \limits_{n=1}^m A_n$ is countable
        \item If $A_n$ is a countable $\forall n \in \N$, then $\bigcup \limits_{n=1}^\infty A_n$ is countable
    \end{enumerate}
\end{theorem}

\subsection{Cantor's theorem}

\subsubsection{Cantor's diagonalization method}

\begin{theorem}
    The open interval $(0,1) = \{ x \in \R: 0 < x < 1 \}$ is uncountable.    
\end{theorem}

\begin{proof}
    For contradiction, assume there exists $f: \N \to (0,1)$ bijective. Next, for each $m \in \N, \exists f(m) \in (0,1)$ we write the decimal representation $f(m) = 0.a_{m,1}a_{m,2}...$. Next, define $b \in (0,1)$ such that $b = 0.b_1b_2...$ with each digit following

    \begin{equation}
        b_n = \begin{cases}
            2 \text{ if } a_{n,n} \ne 2 \\
            3 \text{ if } a_{n,n} = 2
        \end{cases}
    \end{equation}

    Since $b_n \neq a_{n,n} \forall n \in \N$ there exists $b \in (0,1)$ such that $\nexists n \in \N: f(n) = b$ and $f(m)$ is not surjective.
\end{proof}

\begin{corollary}
    The set of real numbers, $\R$, is uncountable.
\end{corollary}

\subsubsection{Power sets and Cantor's theorem}

Given a set $A$, the power set $\mathcal{P}(A)$ refers to the collection of all subsets of $A$.

\begin{theorem}[Cantor's theorem]
    Given a set $A$, there is no function $f: A \to \mathcal{P}(A)$ which is onto.    
\end{theorem}

\begin{proof}
    Assume $f: A \to \mathcal{P}(A)$ is bijective. We prove $f$ cannot be surjective by finding a subset $B \subseteq A$ that is not equal to $f(a)$ for any $a \in A$. Take $B = \{a \in A: a \notin f(a)\}$. If $f$ is surjective, then $B = f(a')$ for some $a' \in A$. However: \\
    \begin{enumerate}
        \item If $a' \in B$ then $a' \notin f(a')$. However, since $B = f(a')$ this implies $a' \notin B$
        \item Else, if $a' \notin B$ then $a' \in f(a') = B$, which is again a contradiction. So, there is no function $f: A \to P(A)$ which is onto.    
    \end{enumerate}
\end{proof}

\begin{theorem}
    $|\N| < |\mathcal{P}(\N)| < |\mathcal{P}(\mathcal{P}(\N))| < ...$
\end{theorem}

Informally, there exists an infinite number of infinitudes.

\begin{theorem}
    If $|A| = n$ then $\mathcal{P}(A) = 2^n$.
\end{theorem}

\begin{corollary}
    $\forall n \in \N \cup \{0\}, n < 2^n$.
\end{corollary}

\section{Epilogue}

Cardinality allows us to create an equivalence relation between sets. In this sense, $\N, \Z, \mathbb{Q}$ are grouped together and are called countable sets. On the other hand, $\R, (a,b), P(\N)$ are uncountable, and belong to a separate group.


Because of the importance of the countable sets, it is usual to denote $\aleph_0 = |\N|$. In terms of cardinal numbers, if $|X| < \aleph_0$ then $X$ is finite. In this way, $\aleph_0$ is the smallest infinite cardinal number. The cardinality of $\R$ also deserves its special designation $\bm c = |\R| = |(0,1)|$. Hence, $\aleph_0 < \bm{c}$.

From this point, one possible question to ask is: ``is there a set $A \subseteq \R: \aleph_0 < |A| < \bm c$''?

Cantor believed there was no such set, leading to the ``continuum hypothesis'' i.e.~$\nexists A \subseteq \R: \aleph_o < |A| < \bm c$. In 1940, Kurt Gödel showed there was no way to disprove this hypothesis from the axioms of set theory. Latter, in 1964 Paul Cohen showed it was also impossible to prove this conjecture. Hence, the problem of the continuum hypothesis is undecidable.