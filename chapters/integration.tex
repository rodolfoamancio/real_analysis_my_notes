\chapter{Integration}

\section{The Riemann Integral}

The Riemann integral is the first rigorous theory of `area', and it is the inverse of differentiation. However, it is not a complete theory of `area', the problems it fails to solve are addressed Lebesque integration. \\

The Riemann integral will be defined over a partition of an interval:

\begin{definition}[Partition]
    A partition $\underline{x}$ of $[a,b]$ is a set $\underline{x} = \{a = x_0 < x_1 < ... < x_n = b \}$. The norm of $\underline{x}$ is denoted by $||\underline{x}||$ and defined by $||\underline{x}||:= \max \{x_1 - x_0, x_2 - x_1, ..., x_n - x_{n-1} \}$.
\end{definition}

\begin{definition}[Tag]
    If $\underline{x}$ is a partition, the tag of $\underline{x}$ is defined as a finite set $\underline{\xi} = \{ \xi_1, \xi_2, ..., \xi_n\}$ such that $a = x_0 \leq \xi_1 \leq x_1 \leq \xi_2 \leq x_2 \leq ... \leq x_{n-1} \leq \xi_n \leq x_n = b$. The pair $(\underline{x}, \underline{\xi})$ is called tagged partition.
\end{definition}

\begin{definition}[Riemann sum]
    Consider $f \in \mathcal{C}([a,b])$. Then, the Riemann sum corresponding to the tagged partition $(\underline{x}, \underline{\xi})$ is the number:
    \begin{equation}
        S_f(\underline{x}, \underline{\xi}):= \sum \limits_{k=1}^n f(\xi_k)(x_k - x_{k-1})
    \end{equation}
\end{definition}

The Riemann sum can be understood as an approximation for the area under the graph of the function $f$. As $||\underline{x}|| \to 0$ we expect this number to converge to $A$, which we interpret as the area under the curve of $f$ on the interval $[a,b]$.

\newpage

\begin{theorem}[Riemann Integral]
    Consider $f \in C([a,b])$. Then, there exists a unique number denoted $\int_{a}^b f(x)\dint x \in \R$ with the property that all sequences of tagged partition $\{(\underline{x}^r, \underline{\xi}^r)\}$ such that $||\underline{x}^r|| \to 0$ then:
    \begin{equation}
        \lim \limits_{r \to \infty} S_f(\underline{x}^r, \underline{\xi}^r) = \int_{a}^b f(x) \dint x
    \end{equation}
\end{theorem}

Next, the derivation of this theorem will be shown. The uniqueness of the integral follows immediately from the uniqueness of the limit, and need not be proven here. In order to prove it, a few tools are presented ahead:

\begin{definition}[Modulus of continuity]
    Consider $f \in \mathcal{C}([a,b])$, and $\eta >0$. Then, the modulus of continuity $\omega_f(\eta)$ is defined as: $\omega_f(\eta) = \sup \{|f(x) - f(y)|:|x-y| \leq \eta\}$.
\end{definition}

\begin{theorem}[Theorem I]
    \begin{equation}
        \lim \limits_{\eta \to 0} \omega_f(\eta) = 0, \forall f \in \mathcal{C}([a,b])
    \end{equation}
    Putting it into words: for any continuous function $f \in \mathcal{C}([a,b])$, for all $\varepsilon > 0$, there exists $\delta > 0$ such that $\forall \eta < \delta, \omega_f(\eta)<\varepsilon$.
\end{theorem}

\begin{proof}
    Consider $\varepsilon > 0$. Since $f \in \mathcal{C}([a,b])$ then $f$ is uniformly continuous on $[a,b]$. Thus, $\exists \delta > 0$ such that $|f(x) - f(y)| < \varepsilon/2, \forall |x-y| < \delta$. Take $\eta < \delta$, then if $|x-y| \leq \eta < \delta$ we have $|f(x) - f(y)| < \varepsilon/2$. So, $\varepsilon/2$ is an upper bound for $\{|f(x) - f(y)|:|x-y|\leq \eta\}$. Therefore, $\omega_f(\eta) \leq \varepsilon/2 < \varepsilon$.
\end{proof}

\begin{theorem}[Theorem II]
    If $(\underline{x}, \underline{\xi})$ and $(\underline{x}', \underline{\xi}')$ are tagged partitions of $[a,b]$ such that $\underline{x} \subseteq \underline{x}'$, then if $f \in \mathcal{C}([a,b])$ we have $|S_f(\underline{x}, \underline{\xi} - S_f(\underline{x}', \underline{\xi}') \leq \omega_f(||\underline{x}||)(b-a)$.
\end{theorem}

\begin{definition}[Refinement]
    If $(\underline{x}, \underline{\xi})$ and $(\underline{x}', \underline{\xi}')$ are tagged partitions of $[a,b]$ such that $\underline{x} \subseteq \underline{x}'$, then $\underline{x}$ is a refinement of $\underline{x}'$.
\end{definition}

\begin{proof}
    For $k = 1, ..., n$. let:
    \begin{eqnarray*}
        \underline{y}(k) &=& \{ x_{k-1} = x_l', x_{l+1}', ..., x_m' = x_k\} \\
        \underline{\eta}(k) &=& \{ \xi_{l+1}', \xi_{l+2}', ..., \xi_{m}'\}
    \end{eqnarray*}
    Then,
    \begin{align*}
        |f(\xi_k)(x_k - x_{k-1}) - S_f(\underline{y}(k), \underline{\eta}(k))| &= \left |
            f(\xi_k) - \sum \limits_{j=l+1}^m f(\xi_j')(x_j' -x_{j-1}') 
        \right | \\
        &= 
        \left |
            \sum \limits_{j=l+1}^m (f(\xi_k) - f(\xi_j'))(x_j' -x_{j-1}') 
        \right |
    \end{align*}
    Since $\sum_{j=1}^m x_j' - x_{j-1}' = x_m - x_l' = x_k - x_{k-1}$:
    \begin{align*}
        |f(\xi_k)(x_k - x_{k-1}) - S_f(\underline{y}(k), \underline{\eta}(k))| &\leq \sum \limits_{j=l+1}^m |f(\xi_k) - f(\xi_j')|(x_j' - x_{j-1}') \\
        &\leq \sum \limits_{j=l+1}^m \omega_f(|x_k - x_{k-1}|)(x_j' - x_{j-1}') \\
        &\leq \omega_f(||\underline{x}||)(x_k - x_{k-1})
    \end{align*}
    Therefore,
    \begin{align*}
        |S_f(\underline{x}, \underline{\xi}) - S_f(\underline{x}', \underline{\xi}')| &= 
        \left |
            \sum \limits_{k=1}^m (f(\xi_k)(x_k - x_{k-1}) - S_j(\underline{y}(k), \underline{\eta}(k)))
        \right | \\
        & \leq
        \sum \limits_{k=1}^m |f(\xi_k)(x_k - x_{k-1}) - S_f(\underline{y}(k), \underline{\eta}(k))| \\
        & \leq \omega_f(||\underline{x}||)\sum \limits_{k=1}^n x_k - x_{k-1} \\
        &= \omega_f(||\underline{x}||)(b-a)
    \end{align*}
\end{proof}

\begin{theorem}[Theorem III]
    If $(\underline{x}, \underline{\xi})$ and $(\underline{x}', \underline{\xi}')$ are two tagged partitions of $[a,b]$, and $f \in \mathcal{C}([a,b])$, then $|S_f(\underline{x}, \underline{\xi}) - S_f(\underline{x}', \underline{\xi}')| \leq (\omega_f(||\underline{x}||) + \omega_f(||\underline{x}'||))(b-a)$.
\end{theorem}

\begin{proof}
    Let $\underline{x}'' = \underline{x} \cup \underline{x}'$ and $\underline{\xi}''$ a tag of $\underline{x}''$. Then, by Theorem II, $|S_f(\underline{x}, \underline{\xi}) - S_f(\underline{x}', \underline{\xi}')| \leq |S_f(\underline{x}, \underline{\xi}) - S_f(\underline{x}'', \underline{\xi}'')| - |S_f(\underline{x}', \underline{\xi}') - S_f(\underline{x}'', \underline{\xi}'')| \leq \omega_f(||\underline{x}||)(b-a) + \omega_f(||\underline{x}'||)(b-a)$
\end{proof}

Finally, we can return to the derivation of the Riemann integral, as desired.

\begin{proof}
    Let $\{\underline{y}(r), \underline{\zeta}(r)\}$ be a sequence of tagged partitions with $||\underline{y}(r)|| \to 0$ as $r \to \infty$.
    First, we prove $\{\underline{y}(r), \underline{\zeta}(r)\}$ is Cauchy: Let $\varepsilon > 0$. By theorem I, $\exists \delta>0$ such that $\forall \eta < \delta$,
    \begin{equation*}
        \omega_f(\eta) < \frac{\varepsilon}{2(b-a)}
    \end{equation*}
    Since $||\underline{y}(r)|| \to 0, \exists N_0 \in N$ such that $||\underline{y}(r)|| < \delta, \forall r \geq N_0$. For $N = N_0$ with $r,s \geq N$, $|S_f(\underline{y}(r), \underline{\zeta}(r)) - S_f(\underline{y}(s), \underline{\zeta}(r))| \leq (\omega_f(||\underline{y}(r)||) + \omega(||\underline{y}(s)||))(b-a)$, by Theorem III. Hence,
    \begin{equation*}
        |S_f(\underline{y}(r), \underline{\zeta}(r)) - S_f(\underline{y}(s), \underline{\zeta}(r))| < \left(
            \frac{\varepsilon}{2(b-a)} + \frac{\varepsilon}{2(b-a)}
        \right)(b-a) = \varepsilon
    \end{equation*}
    So, since the sequence is Cauhcy, define $L = \lim_{r \to \infty} S_f(\underline{y}(r), \underline{\zeta}(r))$ which exists. \\
    Next, we prove that $\lim_{r \to \infty} S_f(\underline{y}(r), \underline{\xi}(r)) = L$ for any sequence of partitions $\{(\underline{x}(r), \underline{\xi}(r))\}$ with $||\underline{x}(r)|| \to 0$. \\
    By the triangle inequatlity and Theorem III,
    \begin{align*}
        |S_f(\underline{x}(r), \underline{\xi}(r)) - L| &\leq |S_f(\underline{x}(r), \underline{\xi}(r)) - S_f(\underline{y}(r), \underline{\zeta}(r)| + |S_f(\underline{y}(r), \underline{\zeta}(r)) - L| \\
        &\leq (\omega_f(||\underline{x}(r)||) + \omega_f(||\underline{y}(r)||))(b-a) + |S_f(\underline{y}(r), \underline{\zeta}(r)) - L| \\
        &\to 0
    \end{align*}
    Thus, by the Squeeze theorem, $|S_f(\underline{x}(r), \underline{\xi}(r)) - L| \to 0$ as $r \to \infty$.
\end{proof}

\section{Properties of the Riemann integral}

\begin{theorem}
    If $f, g \in \mathcal{C}([a,b])$ and $\alpha \in \R$, then
    \begin{enumerate}
        \item Linearity
            \begin{equation}
                \int_a^b (\alpha f) \dint x = \alpha \int_a^b f \dint x
            \end{equation}
        \item Additivity
            \begin{equation}
                \int_a^b (f + g)\dint x = \int_a^b f \dint x + \int_a^b g \dint x
            \end{equation}
    \end{enumerate}
\end{theorem}

\begin{proof}
     Let $\{\underline{x}(r), \underline{\xi}(r)\}$ be a sequence of tagged partitions such that $||\underline{x}(r)|| \to 0$. Then,
    \begin{enumerate}
        \item
            \begin{equation*}
                S_{\alpha f} (\underline{x}(r), \underline{\xi}(r))  = \alpha S_f(\underline{x}(r), \underline{\xi}(r))
            \end{equation*}
            So,
            \begin{align*}
                \int_a^b \alpha f \dint x &= \lim \limits_{r \to \infty} S_{\alpha f}(\underline{x}(r), \underline{\xi}(r)) \\
                &= \lim \limits_{r \to \infty} \alpha S_f(\underline{x}(r), \underline{\xi}(r)) \\
                &= \alpha \int_a^b f \dint x
            \end{align*}
        \item 
            \begin{equation*}
                S_{f + g}(\underline{x}(r), \underline{\xi}(r)) = S_f(\underline{x}(r), \underline{\xi}(r)) + S_g(\underline{x}(r), \underline{\xi}(r))
            \end{equation*}
            So,
            \begin{align*}
                \int_a^b (f+g) \dint x &= \lim \limits_{r \to \infty}S_{f + g}(\underline{x}(r), \underline{\xi}(r)) \\
                &= \lim \limits_{r \to \infty}[ S_f(\underline{x}(r), \underline{\xi}(r)) + S_g(\underline{x}(r), \underline{\xi}(r))] \\
                &= \int_a^b f \dint x + \int_a^b g \dint x
            \end{align*}
    \end{enumerate}
\end{proof}

\begin{theorem}
    Consider $f \in \mathcal{C}([a,b])$, and
    \begin{eqnarray*}
        m_f &=& \inf \{ f(x) : a \leq x \leq b \} \in \R \\
        M_f &=& \sup \{ f(x) : a \leq x \leq b \} \in \R \\
    \end{eqnarray*}
    Then,
    \begin{equation*}
        m_f(b-a) \leq \int_a^b f(x)\dint x \leq M_f(b-a)
    \end{equation*}
\end{theorem}

\begin{proof}
    Consider $\{ (\underline{x}(r), \underline{\xi}(r)\}$ a sequence of tagged partitions such that $|| \underline{x}(r) || \to 0$ as $r \to \infty$. Then,
    \begin{equation*}
        S_f(\underline{x}(r), \underline{\xi}(r)) = \sum \limits_{k=1}^n f(\xi_k(r))(x_k(r) - x_{k-1}(r)) \geq m_f \sum \limits_{k=1}^n (x_k(r) - x_{k-1}(r)) = m_f(b-a)
    \end{equation*}
    And,
    \begin{equation*}
        S_f(\underline{x}(r), \underline{\xi}(r)) = \sum \limits_{k=1}^n f(\xi_k(r))(x_k(r) - x_{k-1}(r)) \leq M_f \sum \limits_{k=1}^n (x_k(r) - x_{k-1}(r)) = M_f(b-a)
    \end{equation*}
    Therefore,
    \begin{equation*}
        m_f(b-a) \leq S_f(\underline{x}(r), \underline{\xi}(r)) \leq M_f(b-a), \forall r
    \end{equation*}
    Taking $\lim_{r \to \infty}$:
    \begin{equation*}
        m_f(b-a) \leq \int_a^b f(x) \dint x \leq M_f(b-a)
    \end{equation*}
\end{proof}

\begin{theorem}
    Consider $f, g \in \mathcal{C}([a,b])$ with $f(x) \leq g(x), \forall x \in [a,b]$, then:
    \begin{equation*}
        \int_a^b f(x) \dint x \leq \int_a^b f(x) \dint x
    \end{equation*}
\end{theorem}

\begin{proof}
    Consider $\{\underline{x}(r), \underline{\xi}(r)\}$ a sequence of tagged partitions, such that $||\underline{x}(r)|| \to 0$ with $r \to \infty$. Then,
    \begin{align*}
        S_f(\underline{x}(r), \underline{\xi}(r)) &= \sum \limits_{j=1}^n f(\xi_j(r)(x_j(r) - x_{j-1}(r)) \\
        &\leq \sum \limits_{j=1}^n g(\xi_j(r))(x_j(r) - x_{j-1}(r)) \\
        &=S_g(\underline{x}(r), \underline{\xi}(r))
    \end{align*}
    Taking the limit as $r \to \infty$
    \begin{equation*}
        \int_a^b f(x) \dint x \leq \int_a^b g(x) \dint x
    \end{equation*}
\end{proof}

\begin{theorem}
    Consider $f \in \mathcal{C}([a,b])$, then:
    \begin{equation*}
        \left |
            \int_a^b f(x) \dint x
        \right | \leq
        \int_a^b |f(x)| \dint x
    \end{equation*}
\end{theorem}

\begin{proof}
    Notice $\pm f(x) \leq |f(x)|, \forall x \in [a,b]$, so:
    \begin{equation*}
        \pm \int_a^b f(x) \dint x \leq \int_a^b f(x) \dint x \Longrightarrow -\int_a^b f(x) \dint x \leq \int_a^b f(x) \dint x \leq \int_a^b |f(x)| \dint x 
    \end{equation*}
\end{proof}

\section{Fundamental theorem of calculus}

\begin{theorem}[Fundamental theorem of calculus]
    Consider $f \in \mathcal{C}([a,b])$.
    \begin{enumerate}
        \item If $F:[a,b] \to \R$ is differentiable and $F'(x) = f(x)$, then
            \begin{equation*}
                \int_a^b f(x) \dint x = F(b) - F(a)
            \end{equation*}
        \item The function 
            \begin{equation*}
                G(x) := \int_a^x f(t) \dint t
            \end{equation*}
            is differentiable on $[a,b]$ and
            \begin{equation*}
                \begin{cases}
                    G'(x) = f(x) \\
                    G(a) = 0
                \end{cases}
            \end{equation*} 
    \end{enumerate}
\end{theorem}

\begin{proof}
    Proving each statement:
    \begin{enumerate}
        \item Consider $\{\underline{x}(r)\}$ a sequence of partitions with $||\underline{x}|| \to 0$ as $r \to \infty$. Then, by the Mean Value Theorem, $\exists \xi_j(r) \in [x_{j-1}(r), x_j(r)], \forall r, j$ such that $F(x_j(r)) - F(x_{j-1}(r)) = F'(\xi_j(r))(x_j(r) - x_{j-1}(r)) = f(\xi_j(r))(x_j(r) - x_{j-1}(r))$. Thus,
        \begin{align*}
            \int_a^b f(x) \dint x &= \lim \limits_{r \to \infty} \sum \limits_{j=1}^{n(r)} f(\xi_j(r))(x_j(r) - x_{j-1}(r)) \\
            &= \lim \limits_{r \to \infty} \sum \limits_{j=1}^{n(r)} [F(x_j(r)) - F(x_{j-1}(r))] \\
            &= \lim \limits_{r \to \infty} F(b) - F(a) \\
            &= F(b) - F(a)
        \end{align*}
        \item Consider $c \in [a,b]$ and let $\varepsilon > 0$, since $f$ is continuous at $c$, $\exists \delta > 0$ such that $|f(t) - f(c)| < \varepsilon/2, \forall t \in (c - \delta, c + \delta)$. \\
        Suppose $0 < x - c < \delta$, if $t \in [c,x]$, then $|t-c| \leq |x-c| < \delta$. Thus,
        \begin{align*}
            \left |
                \frac{1}{x-c} \int_c^x f(t) \dint t - f(c)
            \right | &= \left |
                \frac{1}{x-c} \int_c^x f(t) \dint t -  \frac{1}{x-c} \int_c^x f(c) \dint t
            \right | \\
            &= \frac{1}{x-c} \left |
                \int_c^x (f(t) - f(c)) \dint t
            \right | \\
            &\leq \frac{1}{x-c} \int_c^x |f(t) - f(c)| \dint t \\
            &\leq \frac{1}{x-c} \int_c^x \frac{\varepsilon}{2} \dint t \\
            &= \frac{1}{x-c}\frac{\varepsilon}{2}(x-c) \\
            &= \frac{\varepsilon}{2}
        \end{align*}
        A similar result can be found for $0 < c - x < \delta$. Thus,
        \begin{equation*}
            \left |
                \frac{1}{x-c} \left(\int_a^x f(t) \dint t - \int_a^c f(t) \dint t\right) - f(c)
            \right |
            \leq \frac{\varepsilon}{2}
            < \varepsilon, \forall x \in (c - \delta, c + \delta)
        \end{equation*}
        So,
        \begin{equation*}
            G'(c) = \lim \limits_{x \to c} \frac{1}{x-c} \left(\int_a^x f(t) \dint t - \int_a^c f(t) \dint t\right) = f(c), \forall c \in [a,b]
        \end{equation*}
        Hence, $G'(x) = f(x)$.
    \end{enumerate}
\end{proof}

\section{Integration methods}

\begin{theorem}[Integration by parts]
    Suppose $f, g \in C^1([a,b])$. Then,
    \begin{equation*}
        \int_a^b f'(x)g(x) \dint x = (f(b)g(b) - f(a)g(a)) - \int_a^b f(x) g'(x) \dint x
    \end{equation*}
\end{theorem}

\begin{proof}
    From the derivation properties we have: $(f(x)g(x))' = f'(x)g(x) - f(x)g'(x)$. Therefore, by the Fundamental theorem of calculus:
    \begin{equation*}
        (f(b)g(b) - f(a)g(a)) = \int_a^b f'(x)g(x) \dint + \int_a^b f(x)g'(x) \dint x
    \end{equation*}
\end{proof}

\begin{lemma}[Riemman-Lebesgue]
    Consider $f \in \mathcal{C}([-\pi, \pi])$, $f' \in \mathcal{C}([a,b])$ and $f$ even periodic with period $2 \pi$, \emph{i.e.} $f(-\pi) = f(\pi)$. For $n \in \N \cup \{0\}$, define:
    \begin{eqnarray*}
        a_n &=& \frac{1}{\pi} \int_{-\pi}^\pi f(x) \sin(n x) \dint x \\
        b_n &=& \frac{1}{\pi} \int_{-\pi}^\pi f(x) \cos(n x) \dint x \\
    \end{eqnarray*}
    Then,
    \begin{equation*}
        \lim \limits_{n \to \infty} a_n  = \lim \limits_{n \to \infty} b_n  = 0
    \end{equation*}
\end{lemma}

\begin{definition}[Fourier coefficients]
    The terms $a_n$ and $b_n$ in the previous lemma are called \emph{Fourier coefficients} of $f$.
\end{definition}
Returning to the proof of the Riemman-Lebesgue lemma:
\begin{proof}
    Using integration by parts,
    \begin{align*}
        |b_n| &= \frac{1}{\pi} \left |
            \int_{-\pi}^\pi f(x) \cos (nx) \dint x
        \right | \\
        &= \frac{1}{\pi} \left |
            \int_{-\pi}^\pi \left( \frac{1}{n} \sin (nx)\right)' f(x) \dint x
        \right | \\
        &= \left |
            \frac{1}{n}(f(\pi)\sin(n\pi) - f(-\pi)\sin(n(-\pi))) - \frac{1}{n} \int_{-\pi}^\pi \sin(n x) f'(x) \dint x
        \right |
    \end{align*}
    Notice that $\sin(n \pi) = \sin(n(-\pi)) = 0, \forall n \in \N$. Hence,
    \begin{align*}
        |b_n| &\leq \frac{1}{n} \int_{-\pi}^\pi |\sin(nx)||f'(x)| \dint x \\
        &\leq \frac{1}{n} \int_{-\pi}^\pi |f'| \to 0
    \end{align*}
    So, $|b_n| \to 0$. A similar argument holds for $a_n$.
\end{proof}

\begin{theorem}[Change of variables]
    Consider $\phi: [a,b] \to [c,d]$ to be continuously differentiable with $\phi > 0$ on $[a,b], \phi(a) = c$, and $\phi(b) = d$. Then,
    \begin{equation*}
        \int_c^d f(u) \dint u = \int_a^b f(\phi(x))\phi'(x)\dint x
    \end{equation*}
\end{theorem}

\begin{proof}
    Consider $F: [a,b] \to \R$ such that $F'(x) = f(x)$, then $F(\phi(x))' = f(\phi(x))$. Hence, by the Fundamental theorem of calculus,
    \begin{align*}
        \int_a^b f(\phi(x))\phi'(x) \dint x &= \int_a^b F(\phi(x))' \dint x \\
        &= F(\phi(b)) - F(\phi(a))  \\
        &= F(d) - F(c)
    \end{align*}
    Furthermore,
    \begin{equation*}
        \int_c^d f(u) \dint u = \int_c^d F(u)'\dint u = F(d)- F(c)
    \end{equation*}
\end{proof}